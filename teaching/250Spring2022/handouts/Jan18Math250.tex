\documentclass{article}
\usepackage[utf8]{inputenc}
\usepackage{amssymb}

\title{Jan 18 Math 250: Divisibility}
\author{David Zureick-Brown}
\date{January 2022}

\begin{document}

\maketitle

\textbf{Definition.} Let $a$ and $b$ be integers. We say that $a$ divides $b$ if there exists an integer $k$ such that $b =ak$. In this case, we write $a \mid b$. If $a$ does not divide $b$ we write $a \nmid b$.
\\

In general: when we have a definition, say of blah, the definition of `not blah' is the negation of blah. 
\\

What is the definition of `does not divide'? I.e., what is the negation of `there exists an integer $k$ such that $b =ak$'?

Technically... there does not exist an integer $k$ such that $b = ak$. Better: for all integers $k$, $b \not = ak$. Generally, don't just add ` it is not true that' to the beginning to negate something. 

Consider the sentence `this sentence has five words'.`It is not true that this sentence has five words."
\\

What does it mean to prove that $a \mid b$? For example: $3 \mid 6$? I.e., can we solve the equation $6 = 3k$ (with an integer $k$)? Sure; $k = 2$ works.

To give a proof of a statement about existence, just give an example. 
\\

On the other hand, does $3 \mid 7$? Ask: can we solve the equation $7 = 3k$ with an integer $k$? No: the only solution is $k = 7/3$, which is not an integer.
\\

Why be so so careful? Does $0$ divide 3? I.e., can we solve the equation $3 = 0k$? No. $0k$ is always equal to 0. So $0 \nmid 3$

What about: does $3 \mid 0$? Well, can we solve the equation $0 = 3k$ (with an integer $k$)? Sure: $k = 0$ always works.
\newpage

Basic properties of divisibility. Let $a,b,c$ and $d$ be integers. Then the following are true. \begin{enumerate}
    \item If $a \mid b$, then $a \mid -b$.
    \item (Transitivity) If $a \mid b$ and $b \mid c$, then $a \mid c$. 
    \item (Additivity) If $a \mid b$ and $a \mid c$, then $a \mid b + c$.
    \item (2 out of 3 rule) If $a$ divides at least two of $b$, $c$ and $b + c$, then $a$ divides the third.
   \item (2 out of 3 rule) If $a$ divides at least two of $b$, $c$ and $b - c$, then $a$ divides the third.    
\end{enumerate}

Proof of 1. Suppose that $a \mid b$. Then there exists an integer $k$ such that $b = ak$. Then multiplying by $-1$ gives $-b = -ak = a(-k)$. Since $-k$ is an integer, we conclude that $a \mid -b$.

Proof of 2. Suppose that $a \mid b$ and $b \mid c$. Then for some integers $k$ and $l$, $b = ak$ and $c = bl$. Then $c = (ak)l = a(kl)$. Since $kl$ is an integer, we conclude that $a \mid c$.
\\

Prove that for any integer $n$, $n(n+1)$ is even.
\\

Proof by `cases'. There are two cases: either $n$ is even, or $n$ is odd. If $n$ is even, then $2 \mid n$. Also, $n \mid n(n+1)$. By transitivity, $2 \mid n(n+1)$. Thus $n(n+1)$ is even. 

If $n$ is odd, then $n = 2k+1$ for some integer $k$. Thus $n+1$ = $2k+1 +1 = 2k+2 = 2(k+1)$. Thus $n+1$ is even, so $2 \mid n+1$. Also, $n+1 \mid n(n+1)$. Thus, by transitivity, $2 \mid n(n+1)$ and thus $n(n+1)$ is even. We conclude that, in both cases, $n(n+1)$ is even.
\\

\textbf{Division algorithm}. Let $a$ and $b$ be integers. Then there exist integers $q$ and $r$ such that $a = bq + r$ and $0 \leq r < b$.
\\

This is something that needs to be proved. We won't prove it, and I will have you look at chapter 5 for a proof.
\\ 

HW hint: for all integers $n$, $3 \mid n(n+1)(n+2)$

By the division algorithm, there exist integers $q$ and $r$ such that $n = 3q + r$, where $r = 0, 1,$ or $2$.
\\

Prove that for any integers $a,b$, $2 \mid ab(a-b)$. 
\\

Proof. There are 3 cases: $a$ is even, $b$ is even, or $a$ and $b$ are both odd. If $a$ is even, then by transitivity, $ab(a-b)$ is also even.
If $b$ is even, then by transitivity, $ab(a-b)$ is also even. Finally: if $a$ and $b$ are both odd, then $a-b$ is even, so again by transitivity $ab(a-b)$ is even. We conclude that in each case $ab(a-b)$ is even.






\end{document}
