\documentclass[12pt]{article}
\textwidth=7in \textheight=9.5in \topmargin=-1in \headheight=0in \headsep=.5in \hoffset -.85in
\usepackage{ulem, url, hyperref}
\hypersetup{colorlinks=true, linkcolor=blue, filecolor=magenta, urlcolor=cyan}
\pagestyle{empty}

\begin{document}
\begin{center}
{\Large \textsc{Math 220: Mathematical Reasoning and Proof}}
\end{center}
\begin{center}
  Spring 2024
  \\
  {\color{red}Syllabus IN PROGRESS}  
\end{center}

\hrule \smallskip

\noindent \begin{tabular}{@{}llcccll}
\textbf{Instructor:} & David Zureick-Brown (``DZB'') & & & & \textbf{Office:} & 502 Seeley Mudd \\
 \textbf{Email:} & \href{mailto: dzureickbrown@amherst.edu}{dzureickbrown@amherst.edu} & & & & \textbf{Office hours:} & Thursdays 2:30-3:30 (in person)\\
 \textbf{Time:} & TuTh 10 -- 11:15am & & & & &Wednesdays 2:30-3:30 (via zoom) \\
 \textbf{Place:} & 014 Seeley Mudd & & & & & Wednesdays 3:30-5:30 (in person)
\end{tabular}



\smallskip \hrule \medskip

\noindent\textbf{Textbook:} (This will change)
\href{https://www.amazon.com/Introduction-Abstract-Mathematics-Robert-Bond/dp/1577665392}{``An Introduction to Abstract Mathematics"}, Bond and Keane

\noindent\textbf{Course website:} \url{https://dmzb.github.io/teaching/2024Spring220/}
\medskip

\noindent\textbf{Zoom link:} \url{https://emory.zoom.us/j/97540680666}

\smallskip \hrule \medskip

\noindent \begin{tabular}{@{}lllllll}

\textbf{Math Fellows:} & Gillian Campbell & \href{mailto: gcampbell24@amherst.edu}{gcampbell24@amherst.edu} & Mon &7-7:30 & Thu & 6-7:30 \\
 & &   & SM & 204 & SM & 006\\ 
                       & Kathy Xing & \href{mailto: kxing24@amherst.edu}{kxing24@amherst.edu} & Sun & 7:30-9 & Thu & 7:30-9 \\
 & &   & SM & 204 & SM & 006\\ 

\textbf{Q Center:}& ??? & \href{mailto: atanguay@amherst.edu}{atanguay@amherst.edu}  &Tue &1-4 & Wed &9-11 (Zoom) \\
&&Seeley Mudd 208&Thu&1-4&Fri&12-2
\end{tabular}
\smallskip \hrule \medskip          
\noindent\textbf{Q Center Zoom link}:
\begin{center}
  \url{https://amherstcollege.zoom.us/j/94446399260?pwd=U2FoM1NscWlUM0h5UkROM0tidUVKZz09}
\end{center}

\noindent\textbf{Tanguay appointment link}: \url{https://calendly.com/atanguay-qcenter}
\medskip

\noindent\textbf{QCenter website}: \url{https://www.amherst.edu/academiclife/support/moss_quantitative_center}

\medskip \hrule \medskip

\noindent\textbf{Prerequisites:} Instructor permission.
\smallskip
% \vspace*{.15in}
% \noindent\rule{18cm}{0.4pt}
% \vspace*{.15in}

\noindent \textbf{Learning objectives}: we will cover the following topics.

\begin{itemize}
\item Logic -- statements, negation, converse, contrapositive, quantifiers. \vspace{-5pt}
\item Mathematical prose and rigor -- how to write mathematics
 correctly and in complete sentences.\vspace{-5pt}
\item Techniques of proof -- ``direct'' proofs, proof by contradiction, induction, proof by ``cases''. \vspace{-5pt}
\item Sets, relations, functions -- the building blocks of mathematics. \vspace{-5pt}
\item Some additional topics (e.g., cardinality, different sizes of infinity).
% Numbers (what they are, divisibility transcendental numbers vs. algebraic
% numbers (i.e., the difference between $\pi$ and $2^{(1/2)}$)) \vspace{-5pt}
% % non-constructive proofs, "consider a minimal X" proofs, pigeonhole
% % principle
\end{itemize}

\smallskip \hrule \medskip

\noindent \textbf{Homework} is \textbf{due \underline{Fridays at 12:55pm}}, via \underline{Gradescope}. The assignment should be submitted as a \underline{single file}. Please be kind to our dear graders and take care to make the assignment legible.
\smallskip

\noindent See the document here
\begin{center}
 \url{https://dmzb.github.io/teaching/2024Spring220/assignments-math-220.pdf}
\end{center}
for a list of all assigned work and a weekly breakdown of the course content.
\medskip

\noindent\textbf{Grading:}
Your grade will consist of the following. Your lowest weekly assignment will be dropped.

\begin{center}
 \begin{tabular}{|l|l|}
 \hline
 Weekly Homework Assignments & $40\%$ \\
 \hline
 2 midterms and a comprehensive final exam & $60\% = $ 	 \\
 Best exam & $30\% + $ 	 \\
 Second best exam& $20\% + $ 	 \\
 Third best exam & $10\% + $ 	 \\
 \hline
 \end{tabular}
\end{center}

\vspace*{.15in}
\noindent\textbf{Grade scale:}
A lower bound on your final grade is given by the following table.

\begin{center}
 \begin{tabular}{|ll|ll|ll|ll|l|}
\hline
A &\hspace{-3 pt}\hspace{-7 pt}= 93-100 &B+ &\hspace{-7 pt}= 87-90 &C+ &\hspace{-7 pt}= 77-80 & D+ &\hspace{-7 pt}= 67-70 & F = 0-63\\
A- &\hspace{-3 pt}\hspace{-7 pt}= 90-93 &B &\hspace{-7 pt}= 83-87 & C &\hspace{-7 pt}= 73-77 & D &\hspace{-7 pt}= 63-67 &\\
 & &B- &\hspace{-7 pt}= 80-83 &C- &\hspace{-7 pt}= 70-73 & & &\\
 \hline
 \end{tabular}
\end{center}

\newpage
\vspace*{.15in}
\noindent\textbf{Typical rubric:}
Proofs will typically be graded on the following rubric (out of 10 points).

\begin{center}
 \begin{tabular}{|l|l|}
 \hline
 10& Flawless\\
 9& Basically correct, but not literally 100\% correct\\
 7& Mostly correct, but with at least one error\\
 5& Numerous errors\\
 2& Proof contains a fundamental misunderstanding\\
 0& No part of the proof was correct\\
 \hline
 \end{tabular}
\end{center}

\noindent\textbf{Assignment and exam dates:}

\begin{center}
 \begin{tabular}{|l|l|}
 \hline
 Weekly Assignments & Generally due Fridays at 12:55pm via Gradescope.\\
 \hline
 Midterm 1 & Friday, October 13 \\
 Midterm 2 & Friday, November 17 \\ 
   Final Exam & Monday, December 18, 2-5pm in the usual room (SMUD 204)	 \\
 % Final Exam & Day December ???, 3:00pm - 5:30pm	 \\
 \hline
 \end{tabular}
\end{center}

\noindent Exam problems will be extremely similar to the (easier and medium difficulty) homework problems.
\smallskip

\noindent If you have any conflict with these test dates, PLEASE let me know at least \emph{two} weeks in advance.

\medskip \hrule \medskip



\noindent{\textbf{Getting help}}
\begin{enumerate}
\item[]{\bf Office Hours:} Please stop by to see me (in Seeley Mudd 502) during my scheduled office hours; you can stop by unannounced during these times! If you have scheduling conflicts with my office hours then you are also welcome to make appointments to see me (outside of my regularly scheduled office hours) at a time which is mutually convenient. To schedule an appointment simply send me an email! (In which case, please include your availability in your message.)
\item[] {\bf Math Fellows:} Visit our TAs' office hours, too (see above).
\item[] {\bf The QCenter:} \textbf{Allison Tanguay} of the the Moss Quantitative Center offers Math~220 help, including both appointments and unscheduled drop-ins.
\item[] {\bf Tutoring:} If you need regular one-on-one help from a tutor, we can (probably) help to set that up. To do so, please send me an email.
\end{enumerate}

\smallskip \hrule \medskip

\noindent\textbf{Rewrites} will be allowed (and encouraged) on weekly graded assignments;
 and students can recover up to half of the missed points. Rewrites
 are to be submitted through Gradescope. You may rewrite a problem
 multiple times, and you may resubmit a rewrite as late as you like
 (including right before the final exam).

\medskip

\noindent When you submit a rewrite, please make it clear which problems you are rewriting.

\bigskip
\noindent\textbf{Late submissions}. Any assignment submitted after the due date will be treated as a ``rewrite'' (you can receive up to half credit for the assignment).

\smallskip \hrule \medskip

\noindent\textbf{Honor Code}: Remember that copying another student's work is a violation of the Honor Code and will be treated as such. Please review Amherst College's Honor Code, available \href{https://www.amherst.edu/offices/student-affairs/community-standards/college-standards/honor-code}{here}.

You are free to consult any sources (animate or inanimate) while doing your homework; working in groups is encouraged! % But, if you use anything (or anyone) other than your class notes or the texts listed above, you should say so on your homework -- please state at the end of every problem any sources used.
On the other hand, you are expected to make an honest attempt to do every problem on your own before consulting other sources. Learning and retaining knowledge is a back and forth process of trying problems on your own and asking for help or for a small hint.

\bigskip
\noindent\textbf{Plagiarism}: a good rule of thumb to avoid plagiarism is the following -- when doing the final write up of a problem, do not have any textbooks, web pages, or classmate's write up open in front of you. If you get stuck when writing up an assignment, go back and look again; just make sure that you organize the mathematics in your head before writing a proof rather than copying a solution from some source. \textbf{This is a generous homework policy. Please do not abuse it.}

\bigskip
\noindent \textbf{Calculators, notes, and textbooks are not allowed during exams.}
 If you must leave class during an exam for \textbf{any reason}, please leave all of your belongings (\textbf{including your \sout{handheld supercomputer} phone!}).

\smallskip \hrule \medskip

\noindent\textbf{Inclusivity}: I put great value in welcoming each and every student into the classroom, regardless of their
sex, race, nationality, gender identity, socioeconomic status, ability (intellectual or physical), religious beliefs, or sexual orientation. Each student brings with them to the classroom a unique set of experiences and I expect everyone to contribute to providing an inclusive environment. If, at any time, you experience a situation within this course that you feel challenges your sense of inclusion or accurate assessment of achievement, then please notify me as soon as possible.

\bigskip

\noindent\textbf{Accessibility and accomodations}.
Amherst College complies with the regulations of the Americans with Disabilities Act of 1990 and offers accommodations to students with disabilities. Please \emph{do not hesitate} to ask for accommodations or to contact me about accommodations. (Please also do so as soon as possible.) For more information, please go \href{https://www.amherst.edu/offices/student-affairs/accessibility-services}{here}.

\bigskip
\noindent \textbf{Attendance policy}. Attendance is always optional (except for exams).
% This semester due to the pandemic, some students might be sick or will need to go into isolation or quarantine.
If you are sick, I would prefer that you stay home from class and get notes from a classmate.
% ; I have notes and videos from previous semesters that I can send you.

% \begin{center}
% \begin{tabular}{|r|l|}
% \hline
% Week& content \\
% \hline
% 0& Introduction to the course. \\
% 1& Working from definitions. Simple divisibility proofs. Logic: statements, statement forms,\\
% & truth tables, identity proofs. \\
% 2& Implications. Converse and contrapositive. Proofs of very basic divisibility properties.\\
% 3& Emphasis on problem solving techniques. ``Proof by cases". More difficult divisibility problems. \\
% 4& Proof by contradiction. Unsolvability of equations. Irrationality.\\
% 5& More contradiction. Introduction to induction.\\
% 6& More induction.\\
% 7& Exam. Basics of set theory. Empty set. Proofs with sets.\\
% 8& Operations with Sets. Proofs.\\
% 9& More proofs with sets. De Morgan's laws. Cartesian Product. Power set. \\
% 10& Introduction to functions. Images, domain, codomain. \\
% 11& Exam. More difficult problems with images; preimages.\\
% 12& Injectivity and surjectivity.\\
% 13& Composition and inverse.\\
% 14& Relations.\\
% \hline
% \end{tabular}
% \end{center}

\end{document}