\documentclass[11pt]{article}

\newcommand{\gradescopeCode}{7DP8JD} 
\newcommand{\HWdueTime}{9:55am}
\newcommand{\dueDay}{Thursday}
\newcommand{\courseNumber}{220}
\newcommand{\courseName}{Mathematical Reasoning and Proof}
% \newcommand{\courseName}{Linear algebra}
\newcommand{\season}{Spring}
\newcommand{\thisYear}{2024}
\newcommand{\bookAuthors}{Bond and Keane}
\newcommand{\bookName}{An Introduction to Abstract Mathematics}

\newcommand{\fakesection}[1]{%
  \par\refstepcounter{section}% Increase section counter
  \sectionmark{#1}% Add section mark (header)
  \addcontentsline{toc}{section}{\protect\numberline{\thesection}#1}% Add section to ToC
  % Add more content here, if needed.
}

\newcommand{\fakesectionnocounter}[1]{%
  % \par\refstepcounter{section}% Increase section counter
  \sectionmark{#1}% Add section mark (header)
   \addcontentsline{toc}{section}{\protect\numberline{}#1}% Add section to ToC
  % Add more content here, if needed.
}

\newcommand{\warning}{
\smallskip
\begin{center}
  {\color{red} {\huge\Stopsign} \hspace{1pt}  IN PROGRESS!} {\color{blue}Check back later for the final assignment.} {\color{red} {\huge\Stopsign} }
\end{center}
\smallskip
\AddToShipoutPictureBG*{\AtTextLowerLeft{\llap{\rotatebox[origin=lb]{90}{\large\sffamily\hspace{2.5in}
  {\color{red} {\huge\Stopsign} \hspace{1pt}  IN PROGRESS!} {\color{blue}Check back later for the final assignment.} {\color{red} {\huge\Stopsign} }
      }\quad\rule{0.8pt}{\textheight}\enspace}}}
}

 \expandafter\newcommand\csname dateWeek1\endcsname{Feb 08}     
 \expandafter\newcommand\csname dateWeek2\endcsname{Feb 15}     
 \expandafter\newcommand\csname dateWeek3\endcsname{Feb 22}     
 \expandafter\newcommand\csname dateWeek4\endcsname{Feb 29} 
 \expandafter\newcommand\csname dateWeek5\endcsname{Mar 07}     
 \expandafter\newcommand\csname dateWeek6\endcsname{Mar 14} 
 \expandafter\newcommand\csname dateWeek7\endcsname{Mar 28}     
 \expandafter\newcommand\csname dateWeek8\endcsname{Apr 04}     
 \expandafter\newcommand\csname dateWeek9\endcsname{Apr 11}     
 \expandafter\newcommand\csname dateWeek10\endcsname{Apr 18}     
 \expandafter\newcommand\csname dateWeek11\endcsname{Apr 25}     
 \expandafter\newcommand\csname dateWeek12\endcsname{May 02}     
 \expandafter\newcommand\csname dateWeek13\endcsname{May 07}
 \expandafter\newcommand\csname midterm1Date\endcsname{Mar 14} 
 \expandafter\newcommand\csname midterm2Date\endcsname{Apr 23} 
 \expandafter\newcommand\csname finalDate\endcsname{May ??} 


 \expandafter\newcommand\csname topic1\endcsname{Introduction to course. Mathematical reasoning. Logic}
 \expandafter\newcommand\csname topic2\endcsname{``Direct" proofs and divisibility problems}
 \expandafter\newcommand\csname topic3\endcsname{Proof by contradiction}
 \expandafter\newcommand\csname topic4\endcsname{Induction}
 \expandafter\newcommand\csname topic5\endcsname{Set theory. Basic operations. Proofs with sets} 
 \expandafter\newcommand\csname topic6\endcsname{More sets. DeMorgan’s laws. Cartesian Products. Power sets}
 \expandafter\newcommand\csname topic7\endcsname{Introduction to functions; images and surjectivity}
 \expandafter\newcommand\csname topic8\endcsname{Inverse Image (or “Preimage”)}
 \expandafter\newcommand\csname topic9\endcsname{Injectivity}
 \expandafter\newcommand\csname topic10\endcsname{Composition of functions}
 \expandafter\newcommand\csname topic11\endcsname{Inverse functions}
 \expandafter\newcommand\csname topic12\endcsname{Relations}
 \expandafter\newcommand\csname topic13\endcsname{(Un)countability}

 
\setlength{\oddsidemargin}{-.25in}
\setlength{\evensidemargin}{-.25in}
\setlength{\textwidth}{6.8in}
\setlength{\textheight}{9.3in}
\setlength{\topmargin}{-.8in}
\setlength\parindent{0pt}      

\usepackage{graphicx}
\usepackage{amssymb,amscd,amsfonts,amsbsy}
\usepackage{enumerate}
\usepackage{amsmath,amsthm,amssymb,amsfonts}
\usepackage{mathrsfs}
\usepackage{epsf,epsfig}
\usepackage[usenames,dvipsnames]{xcolor}
\usepackage{minibox}
\usepackage{soul}
\usepackage[inline]{enumitem}
\usepackage{txfonts}

% need these for the warning
\usepackage{marvosym}
\usepackage[normalem]{ulem}
\usepackage{eso-pic, rotating}


\newlist{ienumerate}{enumerate*}{1}
\setlist[ienumerate]{itemjoin = \hspace{1in}, label=(\alph*)}

\usepackage{hyperref} % Required for adding linksand customizing them
\urlstyle{same}
\definecolor{linkcolour}{rgb}{0,0.2,0.6} % Link color
\hypersetup{colorlinks,breaklinks,urlcolor=linkcolour,linkcolor=linkcolour}



\newtheorem{proposition}{Proposition}
\newtheorem{theorem}[proposition]{Theorem}
\newtheorem*{theorem*}{Theorem}

\usepackage{xcolor}
\usepackage{amsthm}
\usepackage{framed}

\colorlet{shadecolor}{red!15}

\usepackage{calc}

\newenvironment{theo}
{\begin{shaded}\begin{theorem*}}
{\end{theorem*}\end{shaded}}

%Auto itemize with columns
\usepackage{etoolbox,refcount}\usepackage{multicol}

\newcounter{countitems}
\newcounter{nextitemizecount}
\newcommand{\setupcountitems}{%
\stepcounter{nextitemizecount}%
\setcounter{countitems}{0}%
\preto\item{\stepcounter{countitems}}%
}
\makeatletter
\newcommand{\computecountitems}{%
\edef\@currentlabel{\number\c@countitems}%
\label{countitems@\number\numexpr\value{nextitemizecount}-1\relax}%
}
\newcommand{\nextitemizecount}{%
\getrefnumber{countitems@\number\c@nextitemizecount}%
}
\newcommand{\previtemizecount}{%
\getrefnumber{countitems@\number\numexpr\value{nextitemizecount}-1\relax}%
}
\makeatother    
\newenvironment{AutoMultiColItemize}{%
\ifnumcomp{\nextitemizecount}{>}{3}{\begin{multicols}{2}}{}%
\setupcountitems\begin{enumerate}}%
{\end{enumerate}%
\unskip\computecountitems\ifnumcomp{\previtemizecount}{>}{3}{\end{multicols}}{}}
%Auto itemize with columns


%HEADER FOOTER
\usepackage{fancyhdr}
%\fancyhf{}% Clear all headers/footers
\fancyhead[L]{Math 220 (Zureick-Brown)}
 \fancyhead[R]{Homework}
%\fancyhead[R]{Due: 11:59pm on Wed., Sept. 12th}
%\pagenumbering{gobble} 



\begin{document}
\pagestyle{fancy}
\thispagestyle{plain}
\begin{center}
  {\Large Math 220-01: Mathematical Reasoning and Proof \\ Instructor: David Zureick-Brown (``DZB'')\smallskip}
\end{center}
\vskip-.05in

\vskip.25in
\begin{center}
 \noindent\textbf{All assignments}\\
 Last updated: \today\\
 \smallskip
 Gradescope code: \gradescopeCode
\end{center}

\begin{center}
\fbox{
\parbox{\textwidth-65\fboxsep}{
\centering
{\bf Show all work for full credit!}

% \textit{Please write all solutions on a separate sheet of paper.}

\textit{Proofs should be written in full sentences whenever possible.}
%\textit{Be sure to include any sources that you used!}

}}
\end{center}
\tableofcontents

\newpage      
\warning \fakesection{(due \csname dateWeek1\endcsname):  \csname topic1\endcsname} \input{assignments-source/S24-220HW1.tex} \newpage 
\warning \fakesection{(due \csname dateWeek2\endcsname):  \csname topic2\endcsname} \input{assignments-source/S24-220HW2.tex} \newpage
\warning \fakesection{(due \csname dateWeek3\endcsname):  \csname topic3\endcsname} \input{assignments-source/S24-220HW3.tex} \newpage
\warning \fakesection{(due \csname dateWeek4\endcsname):  \csname topic4\endcsname} \input{assignments-source/S24-220HW4.tex} \newpage
\warning \fakesection{(due \csname dateWeek5\endcsname):  \csname topic5\endcsname} \input{assignments-source/S24-220HW5.tex} \newpage
\warning \fakesection{(due \csname dateWeek6\endcsname):  \csname topic6\endcsname} \input{assignments-source/S24-220HW6.tex} \newpage
\warning \fakesectionnocounter{(On \csname midterm1Date\endcsname):  Midterm 1}  \input{assignments-source/S24-220M1.tex} \newpage
\warning \fakesection{(due \csname dateWeek7\endcsname):  \csname topic7\endcsname}  \input{assignments-source/S24-220HW7.tex} \newpage
\warning \fakesection{(due \csname dateWeek8\endcsname):  \csname topic8\endcsname}  \input{assignments-source/S24-220HW8.tex} \newpage
\warning \fakesection{(due \csname dateWeek9\endcsname):  \csname topic9\endcsname}  \input{assignments-source/S24-220HW9.tex} \newpage
\warning \fakesection{(due \csname dateWeek10\endcsname): \csname topic10\endcsname} \input{assignments-source/S24-220HW10.tex} \newpage
\warning \fakesectionnocounter{(On \csname midterm2Date\endcsname):  Midterm 2}  \input{assignments-source/S24-220M2.tex} \newpage
\warning \fakesection{(due \csname dateWeek11\endcsname): \csname topic11\endcsname} \input{assignments-source/S24-220HW11.tex} \newpage
\warning \fakesection{(due \csname dateWeek12\endcsname): \csname topic12\endcsname} \input{assignments-source/S24-220HW12.tex} \newpage
\warning \fakesection{(Not due): \csname topic13\endcsname} \smallskip \input{assignments-source/S24-220HW13.tex} \newpage
\warning \fakesectionnocounter{(On \csname finalDate\endcsname):    Final Exam}  \input{assignments-source/S24-220F.tex} \newpage
% \fakesection{(due \csname dateWeek13\endcsname): \csname topic13\endcsname}  %\fbox{\parbox{\textwidth-138\fboxsep}{\bf IN PROGRESS}} \smallskip \input{assignments-source/S24-220HW11.tex} \newpage

\end{document}

Other potential problems
 \begin{enumerate}
 \item Let $A$ and $B$ be sets. We define the \textbf{symmetric difference of} $A$ and $B$ to be
\[
A \Delta B := \{x | x \in A \cup B \text{ and } x \not \in A \cap B\}.
\]

\begin{enumerate}
\item Prove or disprove: $f(A \Delta B) \subset f(A) \Delta f(B)$.
\item Prove or disprove: $f(A) \Delta f(B) \subset f(A \Delta B)$.
\end{enumerate}
\end{enumerate}