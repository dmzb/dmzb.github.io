\documentclass[12pt]{article}
\textwidth=7in \textheight=9.5in \topmargin=-1in \headheight=0in \headsep=.5in \hoffset  -.85in
\usepackage{ulem, url, hyperref}
\hypersetup{colorlinks=true, linkcolor=blue, filecolor=magenta, urlcolor=cyan}
\pagestyle{empty}

\begin{document}
\begin{center}
{\Large \textsc{Math 220: Mathematical Reasoning and Proof}}
\end{center}
\begin{center}
Fall 2023
\end{center}
% \begin{center}
%   {\bf MATH 220: Mathematical Reasoning and Proof
%     \\ Sept 6 - Dec 13, 2023
%     \\ Section 02 MWF 1-1:50 
%     \\ SMUD ??? 
%   }
% \end{center}


% \rule{6in}{0.4pt}
\hrule
\smallskip
\noindent \begin{tabular}{@{}llcccll}
\textbf{Instructor:}     & David Zureick-Brown & & &  & \textbf{Office:} & 502 Seeley Mudd \\
  \textbf{Email:}        &  \href{mailto: dzureickbrown@amherst.edu}{dzureickbrown@amherst.edu} & & & &  \textbf{Office hours:} & Mondays 3-4 (in person)\\
  \textbf{Time:}        &  MWF 1:00 -- 1:50  & & & &  &Thursdays 2-3 (via zoom) \\
  \textbf{Place:}        &  204 Seeley Mudd & & & &  & Thursdays 3-5 (in person)
\end{tabular}
\smallskip
\hrule

\medskip
\noindent\textbf{Textbook:}
\href{https://www.amazon.com/How-Prove-Structured-Daniel-Velleman/dp/1108439535/}
{``How to Prove It: A Structured Approach,"} 3rd Edition by Daniel J.~Velleman
\medskip

\noindent\textbf{Course website:} \url{https://dmzb.github.io/teaching/220Fall2023/}
\medskip

\hrule
\medskip

\noindent\textbf{Prerequisites:} Instructor permission.
\smallskip
%  \vspace*{.15in}
% \noindent\rule{18cm}{0.4pt}
% \vspace*{.15in}


\noindent \textbf{Learning objectives}: we will cover the following topics.

\begin{itemize}
\item Logic -- statements, negation, converse, contrapositive, quantifiers. \vspace{-5pt}
\item Mathematical prose and rigor -- how to write mathematics
  correctly and in complete sentences.\vspace{-5pt}
\item Techniques of proof -- ``direct'' proofs, proof by contradiction, induction, proof by ``cases''. \vspace{-5pt}
\item Sets, relations, functions -- the building blocks of mathematics. \vspace{-5pt}
\item Some additional topics (e.g., cardinality, different sizes of infinity).
% Numbers (what they are, divisibility transcendental numbers vs. algebraic
%    numbers (i.e., the difference between $\pi$ and $2^{(1/2)}$)) \vspace{-5pt}
%   % non-constructive proofs, "consider a minimal X" proofs, pigeonhole
%   % principle
\end{itemize}


\hrule
\medskip

\noindent \textbf{Homework} is \textbf{due \underline{Fridays at 12:55pm}}, via \underline{Gradescope}. The assignment should be submitted as a \underline{single file}. Please be kind to our dear graders and take care to make the assignment legible.
\smallskip

\noindent See the document here
\begin{center}
  \url{https://dmzb.github.io/teaching/220Fall2023/assignments-math-220.pdf}
\end{center}
for a list of all assigned work and a weekly breakdown of the course content.
\medskip

%  This class will meet 28 times. I will cover roughly one section of
%  our text each class. Some sections will be skipped and many will be
%  covered out of order.
% \\

 % There will be many short in class activities in addition to
 % lecturing. % In fact, there will be an activity every day at the
 % % beginning of class (even on non quiz days), so please show up on time! 




\noindent\textbf{Grading:} 
Your grade will consist of the following. Your lowest weekly assignment will be dropped.

\begin{center}
  \begin{tabular}{|l|l|}
    \hline
    Weekly Homework Assignments & $40\%$ \\
    \hline    
    2 midterms and a comprehensive final exam  & $60\% = $ 	 \\
    Best exam  & $30\% + $ 	 \\
    Second best exam& $20\% + $ 	 \\
    Third best exam & $10\% + $ 	 \\    
    \hline
  \end{tabular}
\end{center}

\vspace*{.15in}
\noindent\textbf{Grade scale:} 
A lower bound on your final grade is given by the following table. 

\begin{center}
  \begin{tabular}{|ll|ll|ll|ll|l|}
\hline
A  &\hspace{-3 pt}\hspace{-7 pt}= 93-100   &B+ &\hspace{-7 pt}= 87-90  &C+ &\hspace{-7 pt}= 77-80 & D+ &\hspace{-7 pt}= 67-70  & F = 0-63\\
A- &\hspace{-3 pt}\hspace{-7 pt}= 90-93    &B  &\hspace{-7 pt}= 83-87  & C &\hspace{-7 pt}= 73-77  & D  &\hspace{-7 pt}= 63-67 &\\
   &                         &B- &\hspace{-7 pt}= 80-83  &C- &\hspace{-7 pt}= 70-73 &    &        &\\
    \hline
  \end{tabular}
\end{center}

\newpage
\vspace*{.15in}
\noindent\textbf{Typical rubric:} 
Proofs will typically be graded on the following rubric (out of 10 points).

\begin{center}
  \begin{tabular}{|l|l|}
    \hline
    10& Flawless\\
    9& Basically correct, but not literally 100\% correct\\
    7& Mostly correct, but with at least one error\\    
    5& Numerous errors\\
    2& Proof contains a fundamental misunderstanding\\
    1& Blank\\    
    0& No part of the proof was correct\\
    \hline
  \end{tabular}
\end{center}



\noindent\textbf{Assignment and exam dates:}

\begin{center}
  \begin{tabular}{|l|l|}
    \hline
    Weekly Assignments & Generally due Fridays at 12:55pm via Gradescope.\\
    \hline
    Midterm 1  & Friday, October 13 \\
    Midterm 2  & Friday, November 17 \\         
    Final Exam  & TBA	 \\
    % Final Exam  & Day December ???, 3:00pm - 5:30pm	 \\    
    \hline
  \end{tabular}
\end{center}




% \textbf{Calculators, notes, and textbooks are not allowed in exams or quizzes.}

% \vskip.25in
% \noindent\textbf{Homework}:  There will be homework assigned every week. There
% will be many simple problems, checking your understanding of the
% definitions, that will be collected and graded for completness but not
% correctness. In addition to this, most weeks there will be a number of proofs assigned. You
% are expected to write them up very carefully. I will very carefully grade 3-5 of
% the proofs each week. Homework assignments will typically be worth 40-60 points, depending on the length of the assignment. 
% Only your top 10 homework grades will be counted towards your final grade.

% The homework assignments are available at the course web page, and will be updated after each lecture. \textbf{Please check the webpage (todo: move to Canvas) for changes before beginning the assignment.}
% \\

\vskip.25in
\noindent\textbf{Rewrites} will be allowed (and encouraged) on weekly graded assignments;
  and students can recover up to half of the missed points. Rewrites
  are to be submitted through Gradescope. You may rewrite a problem
  multiple times, and you may resubmit a rewrite as late as you like
  (including right before the final exam).

\medskip

\noindent When you submit a rewrite, please make it clear which problems you are rewriting. 

\vskip.25in
\noindent\textbf{Late submissions}. Any assignment submitted after the due date will be treated as a ``rewrite'' (you can receive up to half credit for the assignment).

% \vskip.25in
% \noindent\textbf{Asynchronous content}: each week, you will have a few short videos to watch, some assigned reading, and a short activity (usually, a quiz or writing assignment) due by Tuesday, 4pm. (This deadline is important, since it will inform the planned synchronous content.)

% \vskip.25in
% \noindent\textbf{Synchronous content}: in the synchronous sessions we will work through problems together, and I will give you feedback on your mathematical writing. The two tools we will use to do this are
% \begin{itemize}
% \item \url{https://awwapp.com/b/}
% \item \url{https://latexbase.com/}
% \end{itemize}
% as well as Zoom. 


\vskip.25in
\noindent\textbf{Honor Code}:  Remember that copying another student's work is a violation of the Honor Code and will be treated as such. Please review Amherst College's Honor Code, available \href{https://www.amherst.edu/offices/student-affairs/community-standards/college-standards/honor-code}{here}.

% If you must leave class during an exam for \textbf{any reason}, please leave all of your belongings (\textbf{including your \sout{handheld supercomputer} phone!}). 

You are free to consult any sources (animate or inanimate) while doing your homework; working in groups is encouraged! % But, if you use anything (or anyone) other than your class notes or the texts listed above, you should say so on your homework -- please state at the end of every problem any sources used. 
On the other hand, you are expected to make an honest attempt to do every problem on your own before consulting other sources. Learning and retaining knowledge is a back and forth process of trying problems on your own and asking for help or for a small hint.
\smallskip

\noindent\textbf{Plagiarism}: a good rule of thumb to avoid plagiarism is the following -- when doing the final write up of a problem, do not have any text books, web pages, or classmate's write up open in front of you. If you get stuck when writing up an assignment, go back and look again; just make sure that you organize the mathematics in your head before writing a proof rather than copying a solution from some source. \textbf{This is a generous homework policy. Please do not abuse it.} 



\vskip.25in
\noindent\textbf{Accessibility and accomodations}. 
Amherst College complies with the regulations of the Americans with Disabilities Act of 1990 and offers accommodations to students with disabilities. Please \emph{do not hesitate} to ask for accommodations or to contact me about accommodations. (Please also do so as soon as possible.)  For more information, please go \href{https://www.amherst.edu/offices/student-affairs/accessibility-services}{here}.

% \vskip.25in
% \noindent\textbf{Health considerations}. At the very first sign of not feeling well, stay at home and reach out for a health consultation. Please consult the \href{https://www.emory.edu/forward/resources/faq/index.html#anchor-health}{campus FAQ} for how to get the health consultation. As you know, Emory does contact tracing if someone has been diagnosed with COVID-19. A close contact is defined as someone you spend more than 15 minutes with, at a distance less than 6 feet, not wearing facial coverings. This typically means your roommates, for example. However, your classmates are not close contacts as long as we are following the personal protective equipment protocols in the classroom: wearing facial coverings, staying six feet apart.

\vskip.25in
\noindent \textbf{Attendance policy}. Attendance is always optional (except for exams).
% This semester due to the pandemic, some students might be sick or will need to go into isolation or quarantine.
If you are sick, I would prefer that you stay home from class and get notes from a classmate.
% ; I have notes and videos from previous semesters that I can send you.

% understand that I will be flexible about assginemnts. Please make sure to email me so that we can discuss your individual circumstances. For students in quarantine who are well, I will provide ways that you can keep up with your schoolwork.  Please also contact me via email if you are in quarantine.


% \vskip.25in
% \noindent\textbf{Overloads}: Ken Mandelberg handles all overloads for the department. The overload form is available at \url{http://www.math.emory.edu/overload-policy.pdf}.


% \vskip.25in
% \noindent\textbf{Additional resources}: todo


% \vskip.25in
% \noindent\textbf{Weekly learning outcomes}: (todo: add section numbers). See table below.n

% \begin{center}
%   \begin{tabular}{|r|l|}
%     \hline
% Week& content \\
%     \hline    
%  0& Introduction to the course.  \\
%     1& Working from definitions. Simple divisibility proofs. Logic: statements, statement forms,\\
%        & truth tables, identity proofs.  \\
%  2& Implications. Converse and contrapositive. Proofs of very basic divisibility properties.\\
%  3& Emphasis on problem solving techniques. ``Proof by cases". More difficult divisibility problems. \\
%  4& Proof by contradiction. Unsolvability of equations. Irrationality.\\
%  5& More contradiction. Introduction to induction.\\
%  6& More induction.\\
%  7& Exam. Basics of set theory. Empty set. Proofs with sets.\\
%  8& Operations with Sets. Proofs.\\
%  9& More proofs with sets. De Morgan's laws. Cartesian Product. Power set. \\
%  10& Introduction to functions. Images, domain, codomain. \\
%  11& Exam. More difficult problems with images; preimages.\\
%  12& Injectivity and surjectivity.\\
%  13& Composition and inverse.\\
%     14& Relations.\\
% \hline    
%   \end{tabular}
% \end{center}
    
\end{document}