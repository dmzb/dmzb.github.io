% replace www.math.emory.edu/~dzb/teaching/250Fall2021/
\documentclass[12pt]{article}
\usepackage{amssymb, latexsym, amsmath, verbatim, multicol}

\pagestyle{empty}
\newcommand{\courseNumber}{220}
\newcommand{\courseName}{Mathematical Reasoning and Proof}
\newcommand{\season}{Fall}
\newcommand{\thisYear}{2023}
\newcommand{\courseTime}{MWF 1 - 1:50} 


 \expandafter\newcommand\csname dateWeek1\endcsname{Sep 15}
 \expandafter\newcommand\csname dateWeek2\endcsname{Sep 22}
 \expandafter\newcommand\csname dateWeek3\endcsname{Sep 29}
 \expandafter\newcommand\csname dateWeek4\endcsname{Oct 06}
 \expandafter\newcommand\csname dateWeek5\endcsname{Oct 13}
 \expandafter\newcommand\csname dateWeek6\endcsname{Oct 20}
 \expandafter\newcommand\csname dateWeek7\endcsname{Oct 27}
 \expandafter\newcommand\csname dateWeek8\endcsname{Nov 03}
 \expandafter\newcommand\csname dateWeek9\endcsname{Nov 10}
 \expandafter\newcommand\csname dateWeek10\endcsname{Nov 17}
 \expandafter\newcommand\csname dateWeek11\endcsname{Dec 01}
 \expandafter\newcommand\csname dateWeek12\endcsname{Dec 08}
 \expandafter\newcommand\csname dateWeek13\endcsname{Dec 13}
 \expandafter\newcommand\csname dateWeek14\endcsname{Dec 22}


% Easy way to fix the margins with a package
\textwidth=7in
\textheight=9.5in
\topmargin=-1in
\headheight=0in
\headsep=.5in
\hoffset -.85in

\setcounter{secnumdepth}{0}

\usepackage{hyperref}
\hypersetup{
 colorlinks=true,
 linkcolor=blue,
 filecolor=magenta,
 urlcolor=cyan,
}

\newcommand{\N}{{\mathbb N}}
\newcommand{\Z}{{\mathbb Z}}
\newcommand{\C}{{\mathbb C}}
\newcommand{\R}{{\mathbb R}}
\newcommand{\Q}{{\mathbb Q}}

\begin{document}
\noindent\rule{18cm}{0.4pt}
\begin{center}
 {\bf MATH \courseNumber, \courseName}\\
 {\bf \courseTime} \\ 
{\bf IN PROGRESS}
\end{center}
\noindent\rule{18cm}{0.4pt}

% \renewcommand{\arraystretch}{2}

\vskip.25in
\begin{center}
 \noindent\textbf{All assignments}\\
 Last updated: \today\\
 \smallskip
 Gradescope code: 7DVWGG
\end{center}
\noindent\rule{18cm}{0.4pt}
\tableofcontents


%***************************************************************************************************************
%***************************************************************************************************************
\newpage
\section[1 (due \csname dateWeek1\endcsname): Introduction to course. Mathematical reasoning. Logic.]{Assignment 1}

\textbf{Topics}: Introduction to the course. Mathematical reasoning. Logic.
\\

\noindent \textbf{Reading}: 1.1, 1.2, 1.5, 2.1, 2.2
\\

\noindent \textbf{Suggested problems} (for extra practice; do not hand in): \href{https://www.math.emory.edu/~dzb/teaching/250Fall2021/handouts/250-H01-logic.pdf}{Handout 1}
\\

% \begin{itemize}
% \item With answers:
%  \begin{itemize}
%  \item 1.1: 
%  \item 1.2: 
%  \item 1.5: 
%  \item 2.1: 
%  \item 2.2:
%  \end{itemize}



% {\bf IN PROGRESS}

\noindent \textbf{Assignment 1, due Friday, \csname dateWeek1\endcsname, via Gradescope
  % TODO
}:
\begin{enumerate}
% \item Add boolean algebra problem, truth table problem (for contrapositive); ``which make mathematical sense'', see V for more

\item Write the negation of each of the following statements.

 \begin{enumerate}
\item All triangles are isosceles.
\item Every door in the building was locked.
\item Some even numbers are multiples of three.
\item Every real number is less than 100.
\item Every integer is positive or negative.
\item If $f$ is a polynomial function, then $f$ is continuous at $0$.
\item If $x^2 > 0$, then $x > 0$.
\item There exists a $y \in \mathbf{R}$ such that $xy = 1$.
\item $(2 > 1)$ and $(\forall x, x^2 > 0)$
\item $\forall \epsilon > 0$, $\exists \delta > 0$ such that if $|x| < \delta$, then $|f(x)| < \epsilon$.
\end{enumerate}

\item Write the converse, contrapositive, and negation of each of the following implications.

 \begin{enumerate}
\item If a quadrilateral is a rectangle, then it has two pairs of parallel sides.
\item $(P \wedge \neg Q) \Rightarrow R$
\item $P  \Rightarrow (R \Rightarrow \forall x, Q(x))$
\end{enumerate}

\item Let $P$ and $Q$ be statements. Write the truth table for
  \begin{enumerate}
  \item $(\neg P) \vee Q$
  \item $(P \wedge (\neg Q)) \Rightarrow Q$
  \end{enumerate}

  
\item Are the statements $(P \vee Q) \wedge R$ and $P \vee (Q \wedge R)$ equivalent? If so, give a proof. If not, explain why by giving a counterexample.

  (Two statement forms are equivalent if they have the same truth tables, and here, a counterexample simply means some choice of truth values for $P,Q,$ and $R$ such that the two statement forms give different outputs.)
\item Let $P$ and $Q$ be statements.
  \begin{enumerate}
  \item Prove that $\neg(P \Rightarrow Q)$ is equivalent to $P \wedge \neg Q$.
  \item Prove that $\neg(P \Rightarrow Q)$ is \emph{not} equivalent to $\neg P \wedge Q$.
  \item Give an example of statements $P$ and $Q$ such that $\neg P \Rightarrow \neg Q$ is true and $\neg(P \Rightarrow Q)$ is false.
  \end{enumerate}

% \item \textbf{TODO} Write a problem where they prove that two statement forms are equivalent using properties rather than truth tables. 

% \item Let $n$ and $m$ be integers. Suppose that $nm$ is even. Is it true that both $n$ and $m$ are even? If so, give a proof. If not, give a counterexample (i.e., give an example that demonstrates that this is false.)

\item Suppose that $n$ is an even integer, and let $m$ be any integer. Prove that $nm$ is even.   
  
\item Suppose that $n$ is an odd integer. Prove that $n^2$ is an odd integer. (Hint: an integer $n$ is odd if and only if there exists an integer $k$ such that $n = 2k+1$.)

\item Prove that if $n^2$ is even, then $n$ is even. (Hint: page 67.)  
  \end{enumerate}

%***************************************************************************************************************
%***************************************************************************************************************
\newpage
\section[2 (due \csname dateWeek2\endcsname): ``Direct'' proofs and divisibility problems.]{Assignment 2}

\textbf{Topics}: ``Direct'' proofs, proof by cases, and divisibility problems. 
\\



\noindent \textbf{Reading}:
\begin{itemize}
\item 3.3, from Definition 3.3.6
\item 6.4, just Definition 6.4.1  
\item see Index to find definitions like prime, etc
\end{itemize}



\noindent \textbf{Suggested problems (do not hand in)}

\begin{enumerate}
% \item With answers: Section 5.3, \#1(a), 4(a), 6(ac)

% \item Without answers: Section 5.3, \#2, 4 (without induction), 5 (without induction)

\item \href{https://www.math.emory.edu/~dzb/teaching/250Fall2021/handouts/250-H02-divisibility.pdf}{Handout 2}
\end{enumerate}


\noindent \textbf{Assignment, due Friday, \csname dateWeek2\endcsname, via Gradescope}:

\begin{enumerate}
\item Suppose that $a \mid b$. Prove that for all $n \in \mathbb{Z}_{> 0}$, $a^n \mid b^n$.
\item Suppose that there exists an integer $n \in \mathbb{Z}_{> 0}$ such that $a \mid b^n$. Is it true that $a \mid b$? Prove or disprove your answer. (For a disproof, please give a counterexample that demonstrates that the statement is false.)
  
\item Prove that for all $a \in \mathbb{Z}$ and for $n \in \mathbb{Z}_{\geq 0}$, $a-1$ divides $a^n -1$.

\item Prove that for all integers $n$, $n$ and $n+1$ have no common divisors other than $\pm 1$.

\item Prove that if $x$ is an integer, then $x^2 + 2$ is not divisible by 4. (Hint: there are two cases: $x$ is even, $x$ is odd. Also, feel free to use basic facts about even or odd, e.g., ``odd + odd = even'', without additional proof.)
\item Prove that the product of three consecutive integers is divisible by 6. (It suffices to prove that it is divisible by 2 and 3 separately.)
\item Show that for all integers $a$ and $b$, 
\[
a^2b^2(a^2 - b^2)
\]
is divisible by 12. (It suffices to prove that it is divisible by 4 and 3 separately.)
\item Find all positive integers $n$ such that $n^2 - 1$ is prime. Prove that your answer is correct.

\end{enumerate}

% %***************************************************************************************************************
% %***************************************************************************************************************
% \newpage
% \section[3 (due \csname dateWeek3\endcsname): Proof by contradiction. Unsolvability. Irrationality.]{Assignment 3}

% \textbf{Topics}: Proof by contradiction. Unsolvability of equations. Irrationality.
% \\

% \noindent \textbf{Reading TODO}:
% \begin{itemize}
% \item Section 1.4, p.~41-42 (stop at Historical Comments)
% \item Section 5.4
% \end{itemize}




% \noindent \textbf{TODO Suggested problems (do not hand in)}

% \begin{enumerate}
% \item Without answers: Section 1.4 \#21
% \item Without answers: Section 5.4 \#6, 7, 10(a), 15, 18,
% \item \href{https://www.math.emory.edu/~dzb/teaching/250Fall2021/handouts/250-H03-contradiction.pdf}{Handout 3}
% \end{enumerate}

% \noindent \textbf{Assignment, due Friday, \csname dateWeek3\endcsname, via Gradescope TODO}:
% \begin{enumerate}
% % \item Prove that the equation $x^2 = 4y + 3$ has no integer solutions. (Hint: there are two cases. Either $x$ is even, or $x$ is odd. Consider each case separately and try to get a contradiction.)
%  % \item Section 5.4, \#13, \#14.
% \item Prove that $2^{1/3}$ is irrational.
% \item Prove that there are no positive integer solutions to the equation $x^2 - y^2 = 10$.
% \item Let $a, b, c$ be integers satisfying $a^2 + b^2 = c^2$. Show that $abc$ must be even. (Harder problem, just for fun: show that $a$ or $b$ must be even.)
% \item Suppose that $a$ and $n$ are integers that are both at least 2. Prove that if $a^n -1$ is prime, then $a = 2$ and $n$ is a prime. (Primes of the form $2^n - 1$ are called Mersenne primes.)
% \end{enumerate}


% %***************************************************************************************************************
% %***************************************************************************************************************
% \newpage
% \section[4 (due \csname dateWeek4\endcsname): Induction.]{Assignment 4}

% \textbf{Topics}: Induction.
% \\

% \noindent \textbf{Reading TODO}: Section 5.2, p.~159-163
% \\

% \noindent \textbf{Fun Video}: Vi Hart; ``Doodling in Math: Spirals, Fibonacci, and Being a Plant''

% \noindent \url{https://www.youtube.com/watch?v=ahXIMUkSXX0}
% \\ 

% \noindent \textbf{TODO Suggested problems (do not hand in)}

% \begin{enumerate}
% \item With answers: Section 5.2 \#1(a), 4(a), 8(ad), 9(a), 29
% \item Without answers: Section 5.2 \#2-9, 13
% \item \href{https://www.math.emory.edu/~dzb/teaching/250Fall2021/handouts/250-H04-induction-warmup.pdf}{Handout 4}
% \item \href{https://www.math.emory.edu/~dzb/teaching/250Fall2021/handouts/250-H05-induction-problems.pdf}{Handout 5}
% \end{enumerate}

% \noindent \textbf{Assignment, due Friday, \csname dateWeek4\endcsname, via Gradescope TODO}:
% \begin{enumerate}
% % \item Prove that for every positive integer $n$,
% % $$1^2 + 2^2 + \ldots n^2 = \frac{n(n+1)(2n+1)}{6}.$$
% \item Prove that for every positive integer $n$,
%  $$1^3 + 2^3 + \cdots +n^3 = \frac{n^2(n+1)^2}{4}.$$ 
% \item Let $a_n$ be defined recursively by $a_1 = 1$ and $a_n = \sqrt{1 + a_{n-1}}$. Prove that for all positive integers $n$, $a_n < 2$.
% \item Prove by induction that if $b_1, b_2, \ldots , b_n$ are even integers, then $b_1 + b_2 + \cdots + b_n$ is even.
%  \item Let $F_1, F_2, F_3, \ldots = 1,1,2,3,5,8,\ldots$ be the Fibonacci sequence. Prove that $F_1^2 + \cdots + F_n^2 = F_nF_{n+1}$.
%  \end{enumerate}


 
% %***************************************************************************************************************
% %***************************************************************************************************************
% \newpage
% \section[5 (due \csname dateWeek5\endcsname): Set theory. Basic operations. Proofs with sets.]{Assignment 5}

% \textbf{Topics}: Basics of set theory. Basic operations. Proofs with sets.
% \\

% \noindent \textbf{Reading TODO}:
% \begin{enumerate}
% \item Section 2.1, p.~49-57; 
% \item Section 2.2, p.~61-65 (stop at DeMorgan's laws)
% \end{enumerate}




% \noindent \textbf{TODO Suggested problems (do not hand in)}

% % Section 2.1, \#7,9,12,14,21,D1 \#13, \#18a, 19a-c, 20e-f, §2.2, \#14, \#26 (2.1) 6,7,9,10,12,13

% \begin{enumerate}
% \item With answers (many of these are calculations; do as many as you need to do to understand the definitions):
%  \begin{enumerate}
%  \item Section 2.1, \#1(adg), 2(adg), 4(adg), 5(a), 7(a), 8(ae), 9(adf), 10(a), 18(acf), 19(ad), 20(ae), 21
%  \item Section 2.2, \#1(adgj), 2(ad), 4(ad), 5(ad), 7(a), 9(ad), 14(a),
% \end{enumerate}

% \item Without answers:
%  \begin{enumerate}
%  \item Section 2.1, 13, 14, 15, 16, 
%  \item Section 2.2, \#1-12
%  \end{enumerate}

% \item \href{https://www.math.emory.edu/~dzb/teaching/250Fall2021/handouts/250-H06-sets-I.pdf}{Handout 6}
% \end{enumerate}

% \noindent \textbf{Assignment 6, due Friday, \csname dateWeek6\endcsname, via Gradescope TODO}:
% \begin{enumerate}
% \item Let $A = \{ n \in \Z | n \text{ is a multiple of } 4\}$ and $B = \{ n \in \Z | n^2 \text{ is a multiple of } 4\}$ 
%  \begin{enumerate}
%  \item Prove or disprove: $A \subseteq B$. 
%  \item Prove or disprove: $B \subseteq A$. 
%  \end{enumerate}

% \item Prove that $A \cup (A \cap B) = A.$
% \item Let $A,B$ and $C$ be sets. 
%  \begin{enumerate}
%  \item Prove that $(A \subseteq C) \wedge (B \subseteq C) \Rightarrow A \cup B \subseteq C$.
%  \item State the contrapositive of part (a).
%  \item State the converse of part (a). Prove or disprove it.
%  \end{enumerate}
% % \item Let $A = \{ n \in \Z | n = 8t+7 \text{ for some } t \in \Z \}$ and $B = \{ n \in \Z | n = 4t + 3 \text{ for some } t \in \Z\}$ 
% % \begin{enumerate}
% % \item Prove of disprove: $A \subseteq B$. 
% % \item Prove of disprove: $B \subseteq A$. 
% % \end{enumerate}
% \item Let $n$ and $m$ be integers. Prove that if $n\Z \subseteq m\Z$ then $m$ divides $n$.
% \end{enumerate}


% % Shpeal about sets (containers, some examples, anything can be a set).
% % How to do proofs with sets. Subset. Definition is an implication. What is your hypothesis? Conclusion? One example, then transitivity. 
% % Empty set. Set containing empty set. Worksheet?

% %***************************************************************************************************************
% %***************************************************************************************************************
% \newpage
% \section[ (October 13) Midterm]{Midterm}

% \textbf{Topics}: Wednesday, October 11 will be an in class Exam review. We will not cover any new material; in class, I will answer whatever questions you have. \textbf{Please show up with questions}.
% \\
% There will be the usual office hours on Thursday, October 12; instead there will be office hours Wednesday, October 18, 2:30-3:30.
% \\

% The exam is on Friday, October 13.
% \\

% \noindent \textbf{Content}: The questions will all be either
% \begin{enumerate}
% \item homework problems,
% \item suggested problems,
% \item problems we worked in class, or
% \item minor variations of one of these.
% \end{enumerate}


% A typical exam will have one or two questions from each week of the course. You can expect problems like the following:
% \begin{itemize}
% \item Negations
% \item Give definitions
% \item Contrapositive  
% \item Contradiction
% \item Induction
% % \item Proofs with sets
% \end{itemize}

% %***************************************************************************************************************
% %***************************************************************************************************************
% \newpage
% \section[6 (due \csname dateWeek6\endcsname): More sets. DeMorgan's laws. Cartesian Products. Power sets.]{Assignment 6}

% \noindent \textbf{Topics}: More proofs with sets. DeMorgan's laws. Cartesian Products. Power sets
% \\

% \noindent \textbf{Reading TODO}:
% \begin{enumerate}
% \item Section 2.2, p.~65-66; 
% \item Section 2.3, p.~72, just the part about
%  power sets.
% \end{enumerate}


% % 2 minutes about DeMorgan's laws. https://www.youtube.com/watch?v=-LW2lGHv0GE 

% \noindent \textbf{TODO Suggested problems (do not hand in)}
 
% \begin{enumerate}
% \item With answers:
%  \begin{enumerate}
%  \item Section 2.2, 13(a), 16(a) 
%  \item Section 2.3, \#1(a), 3, 5(adg),
%  \end{enumerate}

% \item Without answers:
%  \begin{enumerate}
%  \item Section 2.2, 14, 16-19, 21, 23-27
%  \item Section 2.3, \#1(b), 2,4
%  \end{enumerate}

%  \item \href{https://www.math.emory.edu/~dzb/teaching/250Fall2021/handouts/250-H07-sets-II.pdf}{Handout 7}
% \end{enumerate}


% \noindent \textbf{Assignment 7, due Thursday, \csname dateWeek7\endcsname, via Gradescope TODO}:

% \begin{enumerate}
% \item Let $A$ and $B$ be sets. Prove that $(A \cup B) \cap \overline{A} = B - A$.
% \item Let $A$ and $B$ be sets. Prove that $(A \cup B) - (A \cap B) = (A - B) \cup (B - A)$.
% \item Let $A = \{0,1,2\}$. Which of the following statements are true? (No justification is needed.)

%  \begin{enumerate}
%  \item $\{0\} \subseteq P(A)$;\vspace{0.3cm}
%  \item $\{1,2\} \in P(A)$;\vspace{0.3cm}
%  \item $\{1,\{1\}\} \subseteq P(A)$.\\ 
%  \item $\{\{0,1\},\{1\}\} \subseteq P(A)$;\vspace{0.3cm}
%  \item $\emptyset \in P(A)$;\vspace{0.3cm}
%  \item $\emptyset \subseteq P(A)$;\vspace{0.3cm}
%  \item $\{\emptyset\} \in P(A)$. \vspace{0.3cm}
%  \item $\{\emptyset\} \subseteq P(A)$;\vspace{0.3cm}

%  \end{enumerate}
% \item Let $A$ and $B$ be sets. Prove that if $A \subseteq B$, then $P(A) \subseteq P(B)$. State the converse of this and prove or disprove it.

% \end{enumerate}


% %***************************************************************************************************************
% %***************************************************************************************************************
% \newpage
% \section[7 (due \csname dateWeek7\endcsname): Introduction to functions; images and surjectivity.]{Assignment 7}

% % \textbf{NO CLASS March 8, 10; spring break}

% \textbf{Topics}: Introduction to functions; images and surjectivity
% \\

% \noindent \textbf{Reading TODO}:
% \begin{enumerate}
% \item Section 3.1, p.~81-90 (stop at ``Inverse Image'');
% \item Section 3.2, p.~97-100 (stop at Injective Functions).
% \end{enumerate}

% \noindent \textbf{TODO Suggested problems (do not hand in)}

% \begin{enumerate}
% \item With answers:
%   \begin{enumerate}
% \item Section 3.1, \#1(adg), 4(ace), 5(a), 8(a), 10(a), 12(1d)
% \item Section 3.2, \#1(adgj), 2(ad)
% \end{enumerate}  
% \item Without answers:
%   \begin{enumerate}
% \item Section 3.1, 1-4,6-13
% \item Section 3.2, 1-6
% \item \href{https://www.math.emory.edu/~dzb/teaching/250Fall2021/handouts/250-H09-images.pdf}{Handout 9}
%   \end{enumerate}    
% \end{enumerate}


% \noindent \textbf{Assignment 8, due Friday, \csname dateWeek8\endcsname, via Gradescope TODO}:
% \begin{enumerate}

% \item Let $f \colon \mathbf{R} \to \mathbf{R}$ be the function defined by $f(x) = 6x+5$.
%  \begin{enumerate}
%  \item Prove that $f(\mathbf{R}) = \mathbf{R}$.
%  \item Compute $f([1,4])$. Prove your answer.
%  \end{enumerate}
% \item Let $f \colon \mathbf{R} \to \mathbf{R}$ be the function defined by $x^4 + x^2$. 
%  \begin{enumerate}
%  \item Compute the image of $f$. Prove that your answer is correct.
%  \item Compute $f([-1,2])$. Prove that your answer is correct.
% \end{enumerate}

% \item Let $A$ and $B$ be sets and let $X$ and $Y$ be subsets of $A$. Let $f\colon A \to B$ be a function. Prove or disprove each of the following. When giving a disproof, please give an counterexample.
%  \begin{enumerate}

%   \item $f(X \cap Y) \subseteq f(X) \cap f(Y)$.
%   \item $f(X \cap Y) \supset f(X) \cap f(Y)$.
%   \item $f(X) - f(Y) \subseteq f(X - Y)$.
%   \item $f(X) - f(Y) \supset f(X - Y)$.

%  \end{enumerate}

 

% \end{enumerate}

% %***************************************************************************************************************
% %***************************************************************************************************************
% \newpage
% \section[8 (due \csname dateWeek8\endcsname): Inverse Image (or ``Preimage'').]{Assignment 8}

% \textbf{Topics}: Inverse Image (or ``Preimage'').
% \\

% \noindent \textbf{Reading TODO}: Section 3.1, p.~90-92 (stop at the Historical Comments. Or don't.)
% \\



% \noindent \textbf{TODO Suggested problems (do not hand in)}

% \begin{enumerate}
% \item With answers: Section 3.1, \#17(ad), \#18(adg), \#19(a), \#21(a)
% \item Without answers: 17-21
% \item \href{https://www.math.emory.edu/~dzb/teaching/250Fall2021/handouts/250-H10-preimages.pdf}{Handout 10}
% \end{enumerate}

% \noindent Due to a conflict, \textbf{office hours} will be 4:00-5pm on Friday, November 8 (over Zoom). This is after the homework assignment is due; due to this inconvenience, you are welcome to turn in the assignment late (anytime before Thursday, November 10). (Gradescope will mark this as late, but I will not deduct any points.) 
% \\

% \noindent \textbf{Assignment 9, due Friday, \csname dateWeek9\endcsname, via Gradescope TODO}:
% \begin{enumerate}
% \item Let $f \colon \mathbf{R} \to \mathbf{R}$ be the function defined by $f(x) = 3x+1$. 
%  \begin{enumerate}
%  \item Compute $f^{-1}(\{1,5,8\})$ (do not give a proof).
%  \item Compute $f^{-1}(W)$, where $W = (4,\infty)$, and give a proof that your answer is correct.
%  \item Compute $f^{-1}(\mathbf{E})$, where $\mathbf{E}$ is the set of even integers, and give a proof that your answer is correct.
%  \end{enumerate}

% \item Let $f \colon \mathbf{Z} \to \mathbf{Z}$ be the function defined by $f(n) =
%  \begin{cases}
%  \frac{n}{2}, & \text{if $n$ is even} \\
%  2n+4, & \text{if $n$ is odd}.
%  \end{cases}
%  $
 
%  Compute $f^{-1}(\mathbf{E})$. Prove that your answer is correct. (Reminder: $\mathbf{E}$ is the set of even integers.)
% \item Let $A$ and $B$ be sets and let $X$ be a subset of $B$. Let $f\colon A \to B$ be a function. Prove or disprove the following. (For a disproof, please give an explicit counterexample.)
%  \begin{enumerate}
%  \item $X \subseteq f(f^{-1}(X))$.
%  \item $X \supset f(f^{-1}(X))$.
%  \end{enumerate}

% \item Let $A$ and $B$ be sets. Let $S \subseteq A$ and let $T \subseteq B$. Let $f\colon A \to B$ be a function. Prove or disprove the following. (For a disproof, please give an explicit counterexample.)
%  \begin{enumerate}
%  \item $f(S) \subseteq T \Rightarrow S \subseteq f^{-1}(T)$.
%  \item $S \subseteq f^{-1}(T) \Rightarrow f(S) \subseteq T$.
%  \end{enumerate}
 
% \end{enumerate}

% %***************************************************************************************************************
% %***************************************************************************************************************
% \newpage
% \section[9 (due \csname dateWeek9\endcsname): Injectivity.]{Assignment 9}

% \textbf{Topics}: Injectivity.
% \\

% \noindent \textbf{Reading TODO}: Section 3.2, p.~100-105
% \\





% \noindent \textbf{TODO Suggested problems (do not hand in)}

% \begin{enumerate}
% \item With answers: 3.2, \#12(adg), \#13(bd)
% \item Without answers: 3.2 \#9-14, 19(abc)
% \item \href{https://www.math.emory.edu/~dzb/teaching/250Fall2021/handouts/250-H11-injectivity.pdf}{Handout 11}
% \end{enumerate}


% \noindent \textbf{Assignment 10, due Friday, \csname dateWeek10\endcsname, via Gradescope TODO}:
% \begin{enumerate}
% \item Which of the following functions $f \colon \mathbf{R}\to \mathbf{R}$ are injective? If the function is injective, give a proof. If it is not injective, give a counterexample.
%  \begin{enumerate}
%  \item $f(x) = x^4 + x^2$;
%  \item $f(x) = x^3 + x^2$;
%  \item $f(x) =
%  \begin{cases}
%  -x-1, & \text{if $x > 0$} \\
%  x^2, & \text{if $x \leq 0$}.
%  \end{cases}
%  $ 
%  \end{enumerate}
% \item Let $A$ and $B$ be sets and let $X$ and $Y$ be subsets of $A$. Let $f \colon A \to B$ be an injective function. Prove that $f(X \cap Y) = f(X) \cap f(Y)$.
% \item Let $f \colon A \to B$ be a function. Which of the followings statements are equivalent to the statement `$f$ is injective'? (No proof necessary.)
%  \begin{enumerate}
%  \item $f(a) = f(b)$ if $a = b$;
%  \item $f(a) = f(b)$ and $a = b$ for all $a,b \in A$;
%  \item If $a$ and $b$ are in $A$ and $f(a) = f(b)$, then $a = b$;
%  \item If $a$ and $b$ are in $A$ and $a = b$, then $f(a) = f(b)$;
%  \item If $a$ and $b$ are in $A$ and $f(a) \neq f(b)$, then $a \neq b$;
%  \item If $a$ and $b$ are in $A$ and $a \neq b$, then $f(a) \neq f(b)$. 
%  \end{enumerate}
% \item We define a function $f\colon [a,b] \to \mathbf{R}$ to be \textbf{decreasing} if for all $x_1,x_2 \in [a,b]$, if $x_1 < x_2$, then $f(x_1) > f(x_2)$.
%  \begin{enumerate}
%  \item Negate the definition of decreasing.
%  \item Prove that a decreasing function is injective.
%  \end{enumerate}
% \end{enumerate}

% %***************************************************************************************************************
% %***************************************************************************************************************
% \newpage
% \section[10 (due \csname dateWeek10\endcsname): Composition of functions.]{Assignment 10}

% \textbf{Topics}: Composition of functions.
% \\

% \noindent \textbf{Reading TODO}: Section 3.3, p.~110-113
% \\


% \noindent \textbf{TODO Suggested problems (do not hand in)}

% \begin{enumerate}
% \item With answers: 3.3, \#1(a), 2(a), 3(ad), 7(a)
% \item Without answers: 3.3 \#1-7, 9
% \item \href{https://www.math.emory.edu/~dzb/teaching/250Fall2021/handouts/250-H12-compositions.pdf}{Handout 12}
% \end{enumerate}


% \noindent \textbf{Assignment 11, due Friday, \csname dateWeek11\endcsname, via Gradescope TODO}:
% \begin{enumerate}
% \item Let $A, B$ and $C$ be sets and let $f \colon A \to B$ and $g \colon B \to C$ be functions. Prove or disprove each of the following.
%  \begin{enumerate}
%  \item If $g\circ f$ is an injection, then $g$ is an injection.
%  \item If $g\circ f$ is a surjection, then $f$ is a surjection.
%  \item If $g\circ f$ is a surjection, then $g$ is a surjection.
%  \end{enumerate}
% \item Let $A$ and $B$ be sets and let $f \colon A \to B$ and $g \colon B \to A$ be functions. Prove that if $g \circ f$ and $f \circ g$ are bijective, then so are $f$ and $g$.

%  \item Let $f \colon \R \to \R$ and $g \colon \R \to \R$ be functions. Suppose that $f$ and $g$ are both decreasing. Prove that $g \circ f$ is increasing.
%  \end{enumerate}



% %***************************************************************************************************************
% %***************************************************************************************************************
% \newpage
% \section[ (November 17) Midterm]{Midterm}

% \textbf{Topics}: Friday, Nove will be an in class Exam review. We will not cover any new material; in class, I will answer whatever questions you have. \textbf{Please show up with questions}.
% \\
% There will be the usual office hours on Thursday, October 12; instead there will be office hours Wednesday, October 18, 2:30-3:30.
% \\

% The exam is on Friday, October 13.
% \\

% \noindent \textbf{Content}: The questions will all be either
% \begin{enumerate}
% \item homework problems,
% \item suggested problems,
% \item problems we worked in class, or
% \item minor variations of one of these.
% \end{enumerate}


% A typical exam will have one or two questions from each week of the course. You can expect problems like the following:
% \begin{itemize}
% \item Negations
% \item Give definitions
% \item Contrapositive  
% \item Contradiction
% \item Induction
% % \item Proofs with sets
% \end{itemize}

 
% %***************************************************************************************************************
% %***************************************************************************************************************
% \newpage
% \section[11 (due \csname dateWeek11\endcsname): Inverse functions.]{Assignment 11}

% \noindent \textbf{Thanksgiving break is Thursday, November 24 and Friday, November 25};

% \noindent There will no class these days.
% \\

% \noindent Office hours are Wednesday, November 30, and the assignment will be due Thursday, December 1.
% \\

% \noindent \textbf{Topics}: Inverse functions.
% \\

% \noindent \textbf{Reading TODO}: Section 3.3, p.~114-116
% \\

% \noindent \textbf{TODO Suggested problems (do not hand in)}

% \begin{enumerate}
% \item With answers: 3.3 \#10(adgj), 11(a)
% \item Without answers: 3.3 \#10, 12, 14, 15, 17, 18, 19, 22
% \item \href{https://www.math.emory.edu/~dzb/teaching/250Fall2021/handouts/250-H13-inverses.pdf}{Handout 13}
% \end{enumerate}

% \noindent \textbf{Assignment 12, due Thursday, \csname dateWeek12\endcsname, via Gradescope TODO}:
% \begin{enumerate}
% \item Define $f \colon \mathbf{R}-\{1\} \to \mathbf{R} - \{1\}$ by $f(x) = \frac{x+1}{x-1}$. Prove that $f$ is a bijection. Find a formula for the inverse $f^{-1}(x)$, and prove that it is correct.
% \item Let $A, B$ and $C$ be sets and let $f \colon A \to B$ and $g \colon B \to C$ be functions. Prove that if $f$ and $g$ are invertible, then so is $g \circ f$, and prove that $(g \circ f)^{-1} = f^{-1} \circ g^{-1}$.
% \item Let $f \colon \mathbf{R} \to \mathbf{R}$ be the function $f(x) = x^3 + x$. Prove that $f$ is invertible without finding a formula for $f^{-1}$.
% \item Let $A$ and $B$ be sets and let $f \colon A \to B$ be a function. Suppose that $f$ has a \emph{left inverse} $g$; that is, suppose that there exists a function $g \colon B \to A$ such that $g \circ f = id_A$. Prove that $f$ is injective.
% \end{enumerate}

% %***************************************************************************************************************
% %***************************************************************************************************************
% \newpage
% \section[12 (due \csname dateWeek12\endcsname): Relations]{Assignment 12}

% \textbf{Topics}: Relations.
% \\

% \noindent \textbf{Reading TODO}: Section 4.2, p.~139-144 (stop at the proof of Theorem 4.2.6)
% \\



% \noindent \textbf{TODO Suggested problems (do not hand in)}

% \begin{enumerate}
% \item With answers: Section 4.2 \#1(a), 3(ad), 4(a), 5(a), 12(a)
% \item Without answers: Section 4.2 \#1, 3, 4
% \item \href{https://www.math.emory.edu/~dzb/teaching/250Fall2021/handouts/250-H14-equivalence-relations.pdf}{Handout 14}
% \end{enumerate}


% \noindent \textbf{Assignment 13, due Friday, \csname dateWeek13\endcsname, via canvas}:
% \begin{enumerate}
% \item Let $A = \{1,2,3\}$ and define a relation on $A$ by $a \sim b$ if $a + b \neq 3$. Determine whether this relation is reflexive, symmetric, transitive.
% \item Define a relation on $\mathbf{Z}$ given by $a \sim b$ if $a-b$ is divisible by $4$.
%  \begin{enumerate}
%  \item Prove that this is an equivalence relation.
%  \item What integers are in the equivalence class of 18? (No proof necessary.)
%  \item What integers are in the equivalence class of 31? (No proof necessary.) 
%  \item How many distinct equivalence classes are there? What are they? (No proof necessary.)
%  \end{enumerate}
% \item Define a relation on $\mathbf{Z}$ given by $a \sim b$ if $a^2-b^2$ is divisible by $4$.
%  \begin{enumerate}
%  \item Prove that this is an equivalence relation.
%  \item How many distinct equivalence classes are there? What are they? (No proof necessary.)
%  \end{enumerate} 
% \item Let $A$ be a set, and let $P(A)$ be the power set of $A$. Assume that $A$ is not the empty set. Define a relation on $P(A)$ by $X \sim Y$ if $X \subseteq Y$. Is this relation reflexive, symmetric, and/or transitive? In each case give a proof, or disprove with a counterexample. (For a counterexample, give an example of $A$, $X$, and $Y$ that disproves the statement.)
% \end{enumerate}


% % \noindent \textbf{Please note the unusual due date of the homework assignment}.



% %***************************************************************************************************************
% %***************************************************************************************************************
% \newpage
% \section[13 (due \csname dateWeek13\endcsname): Binary Operations and Bijections/Countability]{Assignment 13}

% \textbf{Topics}: Relations.
% \\

% \noindent \textbf{Reading TODO}: Section 4.2, p.~139-144 (stop at the proof of Theorem 4.2.6)
% \\



% \noindent \textbf{TODO Suggested problems (do not hand in)}

% \begin{enumerate}
% \item With answers: Section 4.2 \#1(a), 3(ad), 4(a), 5(a), 12(a)
% \item Without answers: Section 4.2 \#1, 3, 4
% \item \href{https://www.math.emory.edu/~dzb/teaching/250Fall2021/handouts/250-H14-equivalence-relations.pdf}{Handout 14}
% \end{enumerate}


% \noindent \textbf{Assignment 13, due Friday, \csname dateWeek13\endcsname, via canvas}:
% \begin{enumerate}
% \item Let $A = \{1,2,3\}$ and define a relation on $A$ by $a \sim b$ if $a + b \neq 3$. Determine whether this relation is reflexive, symmetric, transitive.
% \item Define a relation on $\mathbf{Z}$ given by $a \sim b$ if $a-b$ is divisible by $4$.
%  \begin{enumerate}
%  \item Prove that this is an equivalence relation.
%  \item What integers are in the equivalence class of 18? (No proof necessary.)
%  \item What integers are in the equivalence class of 31? (No proof necessary.) 
%  \item How many distinct equivalence classes are there? What are they? (No proof necessary.)
%  \end{enumerate}
% \item Define a relation on $\mathbf{Z}$ given by $a \sim b$ if $a^2-b^2$ is divisible by $4$.
%  \begin{enumerate}
%  \item Prove that this is an equivalence relation.
%  \item How many distinct equivalence classes are there? What are they? (No proof necessary.)
%  \end{enumerate} 
% \item Let $A$ be a set, and let $P(A)$ be the power set of $A$. Assume that $A$ is not the empty set. Define a relation on $P(A)$ by $X \sim Y$ if $X \subseteq Y$. Is this relation reflexive, symmetric, and/or transitive? In each case give a proof, or disprove with a counterexample. (For a counterexample, give an example of $A$, $X$, and $Y$ that disproves the statement.)
% \end{enumerate}


% % \noindent \textbf{Please note the unusual due date of the homework assignment}.



% %***************************************************************************************************************
% %***************************************************************************************************************
% \newpage
% \section[Final Exam]{Final Exam}

% \textbf{Final exam} is \textbf{TBA}
% \\

% \noindent The \textbf{last day of class} is Wednesday, December 13. 
% \\

% \noindent There will be \textbf{office hours} on TBA. I will send out a survey to find a time that works for everyone who is planning to attend.% 10:30-11:30 and 6-7, via Zoom. 
% \\

% The final exam will not be comprehensive, and will only cover content introduced after the midterm. Still, while I won't give you problems that are ``just'' about induction, contradiction, negations, etc.~(so for example, I will not ask any irrationality questions) you will still need to use those techniques in some of your proofs.
% \\

% The exam will be, roughly 8-10 questions, with multiple parts. Some questions will be ``prove or disprove''. For disproofs, please write out a counterexample as your disproof.
% \\

% A typical exam will have one or two questions from each week of the course. You can expect a subset of the following:
% \begin{itemize}
% \item Images
% \item preimages
% \item Injectivity 
% \item Surjectivity
% \item Compositions
% \item Invertibility
% \item Relations
% \item Problems from handouts 9-14
% \end{itemize}


% % Typical exams:
% %\ques{30} No definitions.
% % Negate
% % Give definitions
% % Proof by Induction
% % Contradiction
% % Contrapositive
% % Set Identities
% % Abstract functions
% % Injectivity 
% % Surjectivity
% % Invertibility
% % Relations
% % Binary Operations
% % Variants of problems from HO 12-14
% % Problems from HO 15-16


 \end{document}

