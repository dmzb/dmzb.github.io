\documentclass[12pt]{article}
\usepackage{amssymb, latexsym, amsmath, verbatim, multicol}
\usepackage{marvosym}
\usepackage{pgfornament}

\usepackage[normalem]{ulem}
\usepackage{eso-pic, rotating}

\pagestyle{empty}
\newcommand{\courseNumber}{220}
\newcommand{\courseName}{Mathematical Reasoning and Proof}
\newcommand{\season}{Fall}
\newcommand{\thisYear}{2023}
\newcommand{\courseTime}{MWF 1 - 1:50}
\newcommand{\HWdueTime}{12:55pm } 

\newcommand{\warning}{
\smallskip
\begin{center}
  {\color{red} {\huge\Stopsign} \hspace{1pt}  IN PROGRESS!} {\color{blue}Check back later for the final assignment.} {\color{red} {\huge\Stopsign} }
\end{center}
\smallskip
\AddToShipoutPictureBG*{\AtTextLowerLeft{\llap{\rotatebox[origin=lb]{90}{\large\sffamily\hspace{2.5in}
  {\color{red} {\huge\Stopsign} \hspace{1pt}  IN PROGRESS!} {\color{blue}Check back later for the final assignment.} {\color{red} {\huge\Stopsign} }
      }\quad\rule{0.8pt}{\textheight}\enspace}}}
}


 \expandafter\newcommand\csname dateWeek1\endcsname{Sep 15}
 \expandafter\newcommand\csname dateWeek2\endcsname{Sep 22}
 \expandafter\newcommand\csname dateWeek3\endcsname{Sep 29}
 \expandafter\newcommand\csname dateWeek4\endcsname{Oct 06}
 \expandafter\newcommand\csname dateWeek5\endcsname{Oct 13}
 \expandafter\newcommand\csname dateWeek6\endcsname{Oct 20}
 \expandafter\newcommand\csname dateWeek7\endcsname{Oct 27}
 \expandafter\newcommand\csname dateWeek8\endcsname{Nov 03}
 \expandafter\newcommand\csname dateWeek9\endcsname{Nov 10}
 \expandafter\newcommand\csname dateWeek10\endcsname{Nov 17}
 \expandafter\newcommand\csname dateWeek11\endcsname{Dec 01}
 \expandafter\newcommand\csname dateWeek12\endcsname{Dec 08}
 \expandafter\newcommand\csname dateWeek13\endcsname{Dec 13}
 \expandafter\newcommand\csname dateWeek14\endcsname{Dec 22}


% Easy way to fix the margins with a package
\textwidth=7in
\textheight=9.5in
\topmargin=-1in
\headheight=0in
\headsep=.5in
\hoffset -.85in

\setcounter{secnumdepth}{0}

\usepackage{hyperref}
\hypersetup{
 colorlinks=true,
 linkcolor=blue,
 filecolor=magenta,
 urlcolor=cyan,
}

\newcommand{\N}{{\mathbb N}}
\newcommand{\Z}{{\mathbb Z}}
\newcommand{\C}{{\mathbb C}}
\newcommand{\R}{{\mathbb R}}
\newcommand{\Q}{{\mathbb Q}}

\begin{document}
\noindent\rule{18cm}{0.4pt}
\begin{center}
 {\bf MATH \courseNumber, \courseName}\\
 {\bf \courseTime} \\ 
% {\bf IN PROGRESS}
\end{center}
\noindent\rule{18cm}{0.4pt}

% \renewcommand{\arraystretch}{2}

\vskip.25in
\begin{center}
 \noindent\textbf{All assignments}\\
 Last updated: \today\\
 \smallskip
 Gradescope code: 7DVWGG
\end{center}
\noindent\rule{18cm}{0.4pt}
\tableofcontents



%***************************************************************************************************************
%***************************************************************************************************************
\newpage
\section[1 (due \csname dateWeek1\endcsname): Introduction to course. Mathematical reasoning. Logic.]{Assignment 1}

\noindent\textbf{Topics}: Introduction to the course. Mathematical reasoning. Logic.
\\

\noindent \textbf{Reading}: 1.1, 1.2, 1.5, 2.1, 2.2
\\

\noindent \textbf{Suggested problems} (do not hand in; these are just for extra practice): \href{https://www.math.emory.edu/~dzb/teaching/250Fall2021/handouts/250-H01-logic.pdf}{Handout 1}
\\

% \begin{itemize}
% \item With answers:
%  \begin{itemize}
%  \item 1.1: 
%  \item 1.2: 
%  \item 1.5: 
%  \item 2.1: 
%  \item 2.2:
%  \end{itemize}



% {\bf IN PROGRESS}

\noindent \textbf{Assignment 1, due Friday, \csname dateWeek1\endcsname, \HWdueTime via Gradescope}:
\begin{enumerate}
% \item Add boolean algebra problem, truth table problem (for contrapositive); ``which make mathematical sense'', see V for more

\item Write the negation of each of the following statements.

 \begin{enumerate}
\item All triangles are isosceles.
\item Every door in the building was locked.
\item Some even numbers are multiples of three.
\item Every real number is less than 100.
\item Every integer is positive or negative.
\item If $f$ is a polynomial function, then $f$ is continuous at $0$.
\item If $x^2 > 0$, then $x > 0$.
\item There exists a $y \in \mathbf{R}$ such that $xy = 1$.
\item $(2 > 1)$ and $(\forall x, x^2 > 0)$
\item $\forall \epsilon > 0$, $\exists \delta > 0$ such that if $|x| < \delta$, then $|f(x)| < \epsilon$.
\end{enumerate}

\item Write the converse, contrapositive, and negation of each of the following implications.

 \begin{enumerate}
\item If a quadrilateral is a rectangle, then it has two pairs of parallel sides.
\item $(P \wedge \neg Q) \Rightarrow R$
\item $P  \Rightarrow (R \Rightarrow \forall x, Q(x))$
\end{enumerate}

\item Let $P$ and $Q$ be statements. Write the truth table for
  \begin{enumerate}
  \item $(\neg P) \vee Q$
  \item $(P \wedge (\neg Q)) \Rightarrow Q$
  \end{enumerate}

  
\item Are the statements $(P \vee Q) \wedge R$ and $P \vee (Q \wedge R)$ equivalent? If so, give a proof. If not, explain why by giving a counterexample.

  (Two statement forms are equivalent if they have the same truth tables, and here, a counterexample simply means some choice of truth values for $P,Q,$ and $R$ such that the two statement forms give different outputs.)
\item Let $P$ and $Q$ be statements.
  \begin{enumerate}
  \item Prove that $\neg(P \Rightarrow Q)$ is equivalent to $P \wedge \neg Q$.
  \item Prove that $\neg(P \Rightarrow Q)$ is \emph{not} equivalent to $\neg P \wedge Q$.
  \item Give an example of statements $P$ and $Q$ such that $\neg P \Rightarrow \neg Q$ is true and $\neg(P \Rightarrow Q)$ is false.
  \end{enumerate}

% \item \textbf{TODO} Write a problem where they prove that two statement forms are equivalent using properties rather than truth tables. 

% \item Let $n$ and $m$ be integers. Suppose that $nm$ is even. Is it true that both $n$ and $m$ are even? If so, give a proof. If not, give a counterexample (i.e., give an example that demonstrates that this is false.)

\item Suppose that $n$ is an even integer, and let $m$ be any integer. Prove that $nm$ is even.   
  
\item Suppose that $n$ is an odd integer. Prove that $n^2$ is an odd integer. (Hint: an integer $n$ is odd if and only if there exists an integer $k$ such that $n = 2k+1$.)

\item Prove that if $n^2$ is even, then $n$ is even. (Hint: \sout{page 67} contrapositive.)  
  \end{enumerate}

%***************************************************************************************************************
%***************************************************************************************************************
\newpage
\section[2 (due \csname dateWeek2\endcsname): ``Direct'' proofs and divisibility problems.]{Assignment 2}

\noindent\textbf{Topics}: ``Direct'' proofs, proof by cases, and divisibility problems. 
\\



\noindent \textbf{Reading}:
\begin{itemize}
\item 3.3, from Definition 3.3.6
\item 6.4, just Definition 6.4.1  
\item see Index to find definitions like prime, etc
\end{itemize}



\noindent \textbf{Suggested problems (do not hand in; these are just for extra practice)}

\begin{enumerate}
% \item With answers: Section 5.3, \#1(a), 4(a), 6(ac)

% \item Without answers: Section 5.3, \#2, 4 (without induction), 5 (without induction)

\item \href{https://www.math.emory.edu/~dzb/teaching/250Fall2021/handouts/250-H02-divisibility.pdf}{Handout 2}
\end{enumerate}


\noindent \textbf{Assignment, due Friday, \csname dateWeek2\endcsname, \HWdueTime via Gradescope}:

\begin{enumerate}
\item Suppose that $a \mid b$. Prove that for all $n \in \mathbb{Z}_{> 0}$, $a^n \mid b^n$.
\item Suppose that there exists an integer $n \in \mathbb{Z}_{> 0}$ such that $a \mid b^n$. Is it true that $a \mid b$? Prove or disprove your answer. (For a disproof, please give a counterexample that demonstrates that the statement is false.)
  
\item Prove that for all $a \in \mathbb{Z}$ and for $n \in \mathbb{Z}_{\geq 0}$, $a-1$ divides $a^n -1$.

\item Prove that for all integers $n$, $n$ and $n+1$ have no common divisors other than $\pm 1$.

\item Prove that if $x$ is an integer, then $x^2 + 2$ is not divisible by 4. (Hint: there are two cases: $x$ is even, $x$ is odd. Also, feel free to use basic facts about even or odd, e.g., ``odd + odd = even'', without additional proof.)
\item Prove that the product of three consecutive integers is divisible by 6. (It suffices to prove that it is divisible by 2 and 3 separately.)
\item Show that for all integers $a$ and $b$, 
\[
a^2b^2(a^2 - b^2)
\]
is divisible by 12. (It suffices to prove that it is divisible by 4 and 3 separately.)
\item Find all positive integers $n$ such that $n^2 - 1$ is prime. Prove that your answer is correct.

\end{enumerate}

%***************************************************************************************************************
%***************************************************************************************************************
\newpage
\section[3 (due \csname dateWeek3\endcsname): Proof by contradiction]{Assignment 3}

\noindent\textbf{Topics}: Proof by contradiction. Unsolvability of equations. Irrationality.
\\

% \noindent \textbf{Reading TODO}:
% \begin{itemize}
% \item Section 1.4, p.~41-42 (stop at Historical Comments)
% \item Section 5.4
% \end{itemize}


\noindent \textbf{Reading}: 3.2
% \begin{itemize}
% \item Section 1.4, p.~41-42 (stop at Historical Comments)
% \item Section 5.4
% \item  3.2
% \end{itemize}



\noindent \textbf{Suggested problems (do not hand in; these are just for extra practice)}

\begin{enumerate}
% \item Without answers: Section 1.4 \#21
% \item Without answers: Section 5.4 \#6, 7, 10(a), 15, 18,
\item \href{https://www.math.emory.edu/~dzb/teaching/250Fall2021/handouts/250-H03-contradiction.pdf}{Handout 3}
\end{enumerate}

\noindent \textbf{Assignment, due Friday, \csname dateWeek3\endcsname, \HWdueTime via Gradescope}:
\begin{enumerate}
\item Prove that there do not exist integers $a$, and $b$ such that $21a + 30b = 1$.
\item Prove that $2^{1/3}$ is irrational.
\item Suppose that $x$ is a real number such that $0 \leq x \leq \pi/2$. Prove that $\sin x + \cos x \geq 1$. (Hint: at some point in your proof, use that $(\sin x)^2 + (\cos x)^2 = 1$.)
\item Prove that there are no positive integer solutions to the equation $x^2 - y^2 = 10$.
% \item Prove that if $a$ is rational and $b$ is irrational, then $ab$ is irrational.
\item Let $a, b, c$ be integers satisfying $a^2 + b^2 = c^2$. Show that $abc$ must be even. % (Harder problem, just for fun: show that $a$ or $b$ must be even.)
\item Suppose that $a$ and $n$ are integers that are both at least 2. Prove that if $a^n -1$ is prime, then $a = 2$ and $n$ is a prime. (Primes of the form $2^n - 1$ are called Mersenne primes.)
\item Suppose that $a,b \in \mathbb{Z}$. Prove that $a^2 - 4b \neq 2$.

\item Prove that $\log_{10} 7$ is irrational
\end{enumerate}


%***************************************************************************************************************
%***************************************************************************************************************
\newpage
\section[4 (due \csname dateWeek4\endcsname): Induction.]{Assignment 4}

\noindent\textbf{Topics}: Induction.
\\

\noindent \textbf{Reading}: Chapter 6
\\

\noindent \textbf{Fun Video} (optional): Vi Hart; ``Doodling in Math: Spirals, Fibonacci, and Being a Plant''

\noindent \url{https://www.youtube.com/watch?v=ahXIMUkSXX0}
\\ 

\noindent \textbf{Suggested problems (do not hand in; these are just for extra practice)}

\begin{enumerate}
\item \href{https://www.math.emory.edu/~dzb/teaching/250Fall2021/handouts/250-H04-induction-warmup.pdf}{Handout 4}
\item \href{https://www.math.emory.edu/~dzb/teaching/250Fall2021/handouts/250-H05-induction-problems.pdf}{Handout 5}
\end{enumerate}

\noindent \textbf{Assignment, due Friday, \csname dateWeek4\endcsname, \HWdueTime via Gradescope}:
\begin{enumerate}
% \item Prove that for every positive integer $n$,
% $$1^2 + 2^2 + \ldots n^2 = \frac{n(n+1)(2n+1)}{6}.$$
\item Prove that for every positive integer $n$,
 $$1^3 + 2^3 + \cdots +n^3 = \frac{n^2(n+1)^2}{4}.$$ 
\item Let $a_n$ be defined recursively by $a_1 = 1$ and $a_n = \sqrt{1 + a_{n-1}}$. Prove that for all positive integers $n$, $a_n < 2$.
\item Prove by induction that if $b_1, b_2, \ldots , b_n$ are even integers, then $b_1 + b_2 + \cdots + b_n$ is even.
 \item Let $F_1, F_2, F_3, \ldots = 1,1,2,3,5,8,\ldots$ be the Fibonacci sequence. Prove that $F_1^2 + \cdots + F_n^2 = F_nF_{n+1}$.
 % \item Prove that $2^n > n^2$ for all $n \geq 5$.
 \item Prove that $n! > 2^n$ for all $n \geq 4$.
 \item \emph{Bernoulli's inequality}: let $\beta \in \mathbb{R}$ be a real number such that $\beta > -1$ and $\beta \neq 0$. Prove that for all integers $n \geq 2$, $(1 + \beta)^n > 1 + n\beta$.
 \item Prove that for all integers $n \geq 1$,
   \[
1 + \frac{1}{\sqrt{2}} + \cdots + \frac{1}{\sqrt{n}} \geq \sqrt{n}.
     \]
   \item Prove (using induction) that for all integers $n \geq 1$, $2^{2n}-1$ is divisible by 3.
 \end{enumerate}


 
%***************************************************************************************************************
%***************************************************************************************************************
\newpage
\section[5 (due \csname dateWeek5\endcsname): Set theory. Basic operations. Proofs with sets.]{Assignment 5}

\noindent\textbf{Topics}: Basics of set theory. Basic operations. Proofs with sets.
\\

\noindent \textbf{Reading}: 1.3, 1.4, 2.3
% \begin{enumerate}
% \item 
% \end{enumerate}




\noindent \textbf{Suggested problems (do not hand in; these are just for extra practice)}: \href{https://www.math.emory.edu/~dzb/teaching/250Fall2021/handouts/250-H06-sets-I.pdf}{Handout 6}
\\
% Section 2.1, \#7,9,12,14,21,D1 \#13, \#18a, 19a-c, 20e-f, §2.2, \#14, \#26 (2.1) 6,7,9,10,12,13

% \begin{enumerate}
% \item With answers (many of these are calculations; do as many as you need to do to understand the definitions):
%  \begin{enumerate}
%  \item Section 2.1, \#1(adg), 2(adg), 4(adg), 5(a), 7(a), 8(ae), 9(adf), 10(a), 18(acf), 19(ad), 20(ae), 21
%  \item Section 2.2, \#1(adgj), 2(ad), 4(ad), 5(ad), 7(a), 9(ad), 14(a),
% \end{enumerate}

% \item Without answers:
%  \begin{enumerate}
%  \item Section 2.1, 13, 14, 15, 16, 
%  \item Section 2.2, \#1-12
%  \end{enumerate}


% \end{enumerate}
\smallskip
\noindent \textbf{Assignment 5, due Friday, \csname dateWeek5\endcsname, \HWdueTime via Gradescope}:
\begin{enumerate}
\item Let $A = \{ n \in \Z | n \text{ is a multiple of } 4\}$ and $B = \{ n \in \Z | n^2 \text{ is a multiple of } 4\}$ 
 \begin{enumerate}
 \item Prove or disprove: $A \subseteq B$. 
 \item Prove or disprove: $B \subseteq A$. 
 \end{enumerate}

\item Prove that $A \cup (A \cap B) = A.$
\item Let $A,B$ and $C$ be sets. 
 \begin{enumerate}
 \item Prove that $(A \subseteq C) \wedge (B \subseteq C) \Rightarrow A \cup B \subseteq C$.
 \item State the contrapositive of part (a).
 \item State the converse of part (a). Prove or disprove it.
 \end{enumerate}
% \item Let $A = \{ n \in \Z | n = 8t+7 \text{ for some } t \in \Z \}$ and $B = \{ n \in \Z | n = 4t + 3 \text{ for some } t \in \Z\}$ 
% \begin{enumerate}
% \item Prove of disprove: $A \subseteq B$. 
% \item Prove of disprove: $B \subseteq A$. 
% \end{enumerate}
\item Let $n$ and $m$ be integers. Prove that if $n\Z \subseteq m\Z$ then $m$ divides $n$.
\end{enumerate}


%***************************************************************************************************************
%***************************************************************************************************************
\newpage
\section[ (October 13) Midterm]{Midterm study guide}

\noindent\textbf{Topics}: Friday, October 13 will be an in class Exam.
\\

\noindent \textbf{Content}: The questions will all be either
\begin{enumerate}
\item homework problems,
\item suggested problems,
\item problems we worked in class, or
\item minor variations of one of these.
\end{enumerate}

Problems with very long proofs or that involved some unusual trick will not be on the exam.
\smallskip

You are allowed to use any previous problem from class or from the homework (e.g., ``additivity of divisibility'' or ``the 2 out of 3 rule'') on the exam without reproving it, unless otherwise noted on the exam. (E.g., if I ask you to prove ``additivity of divisibility'' on the exam, you will need to prove this using only the definition of divisibility, and I will remind you of this in the statement of the problem.)
\smallskip

A typical exam will have one or two questions from each week of the course. You can expect problems about following:
\begin{itemize}
\item Negations
\item Give definitions (e.g., divides, rational, subset)
\item Direct proofs
\item Proof by contrapositive
\item Divisibility problems
\item Contradiction
\item Induction
\item Proofs with sets. 
\end{itemize}
There will be a negation problem, at least one definition, and around 4-5 problems involving proofs (possibly including ``prove or disprove'' problems).
\smallskip

For sets: I will ask one problem, verbatim, from the homework or from class.
\smallskip

For definitions, I want a definition, in prose (complete sentences), and I want ``just'' the definition, and not any additional facts about the definition. (E.g., if you give the definition of rational, do not include that a rational number can be written in reduced form; that is a fact about rational numbers not part of the definition of rational.)


%***************************************************************************************************************
%***************************************************************************************************************
\newpage
\section[6 (due \csname dateWeek6\endcsname): More sets. DeMorgan's laws. Cartesian Products. Power sets.]{Assignment 6}

\noindent \noindent\textbf{Topics}: More proofs with sets. DeMorgan's laws. Cartesian Products. Power sets
\\

% \noindent \textbf{Reading TODO}:
% \begin{enumerate}
% \item Section 2.2, p.~65-66; 
% \item Section 2.3, p.~72, just the part about
%  power sets.
% \end{enumerate}


% 2 minutes about DeMorgan's laws. https://www.youtube.com/watch?v=-LW2lGHv0GE 

\noindent \textbf{Suggested problems (do not hand in; these are just for extra practice)}
 
\begin{enumerate}
% \item With answers:
%  \begin{enumerate}
%  \item Section 2.2, 13(a), 16(a) 
%  \item Section 2.3, \#1(a), 3, 5(adg),
%  \end{enumerate}

% \item Without answers:
%  \begin{enumerate}
%  \item Section 2.2, 14, 16-19, 21, 23-27
%  \item Section 2.3, \#1(b), 2,4
%  \end{enumerate}

 \item \href{https://www.math.emory.edu/~dzb/teaching/250Fall2021/handouts/250-H07-sets-II.pdf}{Handout 7}
\end{enumerate}
\smallskip

\noindent \textbf{Assignment 6, due Friday, \csname dateWeek6\endcsname, \HWdueTime via Gradescope}:

\begin{enumerate}
\item Recall that $(a,b) = \{x : x \in \mathbb{R} \,|\, a < x < b\}$. Prove or disprove each of the following:
 \begin{enumerate}
 \item $(-1,1) \subseteq (-2,2)$. 
 \item $(-1,2) \subseteq (-2,1)$.
 \end{enumerate}



 \item Let $A,B$ be sets. Prove each of the following:
 \begin{enumerate}
 \item $A \cap B \subseteq A$;
 \item $A \cap \emptyset = \emptyset$;
 \end{enumerate}

\item Prove that if $A \not \subseteq C$ then  $A \not \subseteq B$ or $B \not \subseteq C$.
 
 \item Let $A,B$, and $C$ be sets. Prove or disprove the following. (For a disproof, please give an explicit counterexample; i.e., give an example of sets $A$, $B$ and $C$ demonstrating that the statement is false.)
 \begin{enumerate}
 % \item If $A \subseteq B$, $B \subseteq C$, and $C \subseteq D$, then $A \subseteq D$.
 \item If $A \not \subseteq B$ and $B \not \subseteq C$, then $A \not \subseteq C$.
 \item If $A \subseteq B$ and $B \not \subseteq C$, then $A \not \subseteq C$.
 \end{enumerate}
 
 \item Let $A,B,C$ be sets. Prove each of the following:
 \begin{enumerate}
 \item Suppose that $B \subseteq C$. Prove that $A-C \subseteq A-B$.
 \item $A \subseteq B$ if and only if $A \cap B = A$.
 \end{enumerate}
 
  \item Let $A,B$ and $C$ be sets. Prove or disprove each of the following. (For a disproof, please give an explicit counterexample; i.e., give an example of sets $A$, $B$ and $C$ demonstrating that the statement is false.)
 \begin{enumerate}
\item $(A \cap B ) \cup C = A \cap (B  \cup C)$.
\item $(A \cap B ) \cup C = (A \cup C) \cap (B  \cup C)$  .
 \end{enumerate}

\item Let $A$ and $B$ be sets. Prove that $(A \cup B) - (A \cap B) = (A - B) \cup (B - A)$.
\newpage
\item Let $A = \{0,1,2\}$. Which of the following statements are true? (No justification is needed.)

 \begin{enumerate}
 \item $\{0\} \subseteq P(A)$;\vspace{0.3cm}
 \item $\{1,2\} \in P(A)$;\vspace{0.3cm}
 \item $\{1,\{1\}\} \subseteq P(A)$.\\ 
 \item $\{\{0,1\},\{1\}\} \subseteq P(A)$;\vspace{0.3cm}
 \item $\emptyset \in P(A)$;\vspace{0.3cm}
 \item $\emptyset \subseteq P(A)$;\vspace{0.3cm}
 \item $\{\emptyset\} \in P(A)$. \vspace{0.3cm}
 \item $\{\emptyset\} \subseteq P(A)$;\vspace{0.3cm}

 \end{enumerate}

\end{enumerate}


%***************************************************************************************************************
%***************************************************************************************************************
\newpage
\section[7 (due \csname dateWeek7\endcsname): Introduction to functions; images and surjectivity.]{Assignment 7}

\noindent\textbf{Topics}: Introduction to functions; images and surjectivity
\\

\noindent \textbf{Reading}: Chapter 5, especially 5.1 and 5.5
\\

\noindent \textbf{Suggested problems (do not hand in; these are just for extra practice)} \href{https://www.math.emory.edu/~dzb/teaching/250Fall2021/handouts/250-H09-images.pdf}{Handout 9}
\medskip


\noindent \textbf{Assignment 7, due Friday, \csname dateWeek7\endcsname, \HWdueTime via Gradescope}:
\begin{enumerate}

\item Let $A$ and $B$ be sets. Prove that $(A \cup B) \cap \overline{A} = B - A$.

\item Let $A$ and $B$ be sets. Prove that if $A \subseteq B$, then $P(A) \subseteq P(B)$. State the converse of this and prove or disprove it.  
  
\item Let $f \colon \mathbf{R} \to \mathbf{R}$ be the function defined by $f(x) = 6x+5$.
 \begin{enumerate}
 \item Prove that $f(\mathbf{R}) = \mathbf{R}$.
 \item Compute $f([1,4])$. Prove your answer.
 \end{enumerate}
\item Let $f \colon \mathbf{R} \to \mathbf{R}$ be the function defined by $x^4 + x^2$. 
 \begin{enumerate}
 \item Compute the image of $f$. Prove that your answer is correct.
 \item Compute $f([-1,2])$. Prove that your answer is correct.
\end{enumerate}

\item Let $g \colon \mathbb{R} \to \mathbb{Z}$ be the \textbf{ceiling function} $g(x) = \lceil x \rceil$, defined to be the smallest integer greater than or equal to $x$ (i.e., ``round $x$ up to the nearest integer''; so $g(1.3)  = 2$, and $g(3) = 3$.

  Compute the image of $g$. Prove that your answer is correct.


  
\item Consider the function $\sin \colon \mathbf{R} \to \mathbf{R}$.
  \begin{enumerate}
  \item Compute the image of $\sin$. Prove that your answer is correct.
  \item Compute $\sin([0,\pi/4])$. Prove that your answer is correct.
  \end{enumerate}

\item Let $A$ and $B$ be sets and let $X$ and $Y$ be subsets of $A$. Let $f\colon A \to B$ be a function. Prove or disprove each of the following. When giving a disproof, please give a counterexample.
 \begin{enumerate}

  \item $f(X \cup Y) \subseteq f(X) \cup f(Y)$.
  \item $f(X \cup Y) \supseteq f(X) \cup f(Y)$.


 \end{enumerate}

\item Let $A$ and $B$ be sets and let $X$ and $Y$ be subsets of $A$. Let $f\colon A \to B$ be a function. Prove or disprove each of the following. When giving a disproof, please give an counterexample.
 \begin{enumerate}
  \item $f(X) - f(Y) \subseteq f(X - Y)$.
  \item $f(X) - f(Y) \supseteq f(X - Y)$.

 \end{enumerate}
 
 

\end{enumerate}

%***************************************************************************************************************
%***************************************************************************************************************
\newpage
\section[8 (due \csname dateWeek8\endcsname): Inverse Image (or ``Preimage'').]{Assignment 8}

\noindent\textbf{Topics}: Inverse Image (or ``Preimage'').
\\

\noindent \textbf{Reading}: Chapter 5, especially 5.1 and 5.5
\\



\noindent \textbf{Suggested problems (do not hand in; these are just for extra practice)} \href{https://www.math.emory.edu/~dzb/teaching/250Fall2021/handouts/250-H10-preimages.pdf}{Handout 10}
\\

\noindent \textbf{Assignment 8, due Friday, \csname dateWeek8\endcsname, \HWdueTime via Gradescope}:
\\

\noindent (REMINDER: you are allowed to use the results of previous problem as part of the proof of later problems.)

\begin{enumerate}
\item Let $f \colon \mathbf{R} \to \mathbf{R}$ be the function defined by $f(x) = 3x+1$. 
 \begin{enumerate}
 \item Compute $f^{-1}(\{1,5,8\})$ (do not give a proof).
 \item Compute $f^{-1}(W)$, where $W = (4,\infty)$, and give a proof that your answer is correct.
 \item Compute $f^{-1}(\mathbf{E})$, where $\mathbf{E}$ is the set of even integers, and give a proof that your answer is correct.
 \end{enumerate}

\item Let $f \colon \mathbf{Z} \to \mathbf{Z}$ be the function defined by $f(n) =
 \begin{cases}
 \frac{n}{2}, & \text{if $n$ is even} \\
 2n+4, & \text{if $n$ is odd}.
 \end{cases}
 $
 
 Compute $f^{-1}(\mathbf{E})$. Prove that your answer is correct. (Reminder: $\mathbf{E}$ is the set of even integers.)

\item Let $A$ and $B$ be sets and let $X$ and $Y$ be subsets of $B$. Let $f\colon A \to B$ be a function. Prove or disprove the following. (For a disproof, please give an explicit counterexample.) 
  \begin{enumerate}
  \item $f^{-1}(X \cap Y) \subseteq f^{-1}(X) \cap f^{-1}(Y)$.
  \item $f^{-1}(X \cap Y) \supseteq f^{-1}(X) \cap f^{-1}(Y)$.
  \end{enumerate}


\item Let $A$ and $B$ be sets and let $X$ be a subset of $B$. Let $f\colon A \to B$ be a function. Prove or disprove the following. (For a disproof, please give an explicit counterexample.)
 \begin{enumerate}
 \item $X \subseteq f(f^{-1}(X))$.
 \item $X \supseteq f(f^{-1}(X))$.
 \end{enumerate}

% \item Let $A$ and $B$ be sets and let $X$ be a subset of $B$. Let $f\colon A \to B$ be a function. Prove or disprove the following. (For a disproof, please give an explicit counterexample.)
%  \begin{enumerate}
%  \item $f^{-1}(X - Y) \subseteq f^{-1}(X) - f^{-1}(Y)$.
%  \item $f^{-1}(X - Y) \supseteq f^{-1}(X) - f^{-1}(Y)$.
%  \end{enumerate}
 
\item Let $A$ and $B$ be sets. Let $S \subseteq A$ and let $T \subseteq B$. Let $f\colon A \to B$ be a function. Prove or disprove the following. (For a disproof, please give an explicit counterexample.)
 \begin{enumerate}
 \item $f(S) \subseteq T \Rightarrow S \subseteq f^{-1}(T)$.
 \item $S \subseteq f^{-1}(T) \Rightarrow f(S) \subseteq T$.
 \end{enumerate}

\newpage
\noindent In each of the following problems, let $f\colon A \to A$ be a function (note that the domain and codomain are the same) and suppose that $C \subseteq A$.
 
\item  Prove or disprove the following. (For a disproof, please give an explicit counterexample).
 \begin{enumerate}
 \item $f^{-1}(C)\subseteq  C$;
 \item $C\subseteq  f^{-1}(C)$;
 \item $f(C)\subseteq  C$;
 \item $C\subseteq  f(C)$;
 \end{enumerate}
 


\item Prove or disprove the following (for a disproof, please give an explicit counterexample): $C \subseteq  f^{-1}(C) \iff f(C)\subseteq  C$ . 

\item Prove or disprove the following (for a disproof, please give an explicit counterexample): $f^{-1}(C)\subseteq  C \iff C\subseteq  f(C)$ .
  
\end{enumerate}

%***************************************************************************************************************
%***************************************************************************************************************
\newpage
\section[9 (due \csname dateWeek9\endcsname): Injectivity.]{Assignment 9}

% \warning 

\noindent\textbf{Topics}: Injectivity.
\\

\noindent \textbf{Reading}: Section 5.2
\\

\noindent \textbf{Suggested problems (do not hand in; these are just for extra practice)}:
\href{https://www.math.emory.edu/~dzb/teaching/250Fall2021/handouts/250-H11-injectivity.pdf}{Handout 11}
\\


\noindent \textbf{Assignment 9, due Friday, \csname dateWeek9\endcsname, \HWdueTime via Gradescope}:
\begin{enumerate}
\item Let $A$ and $B$ be sets and let $X$ and $Y$ be subsets of $B$. Let $f\colon A \to B$ be a function. Prove or disprove the following. (For a disproof, please give an explicit counterexample.)
 \begin{enumerate}
 \item $f^{-1}(X - Y) \subseteq f^{-1}(X) - f^{-1}(Y)$.
 \item $f^{-1}(X - Y) \supseteq f^{-1}(X) - f^{-1}(Y)$.
 \end{enumerate} 

\item Let $f \colon A \to B$ be a function. Which of the followings statements are equivalent to the statement `$f$ is injective'? (No proof necessary.)
 \begin{enumerate}
 \item $f(a) = f(b)$ if $a = b$;
 \item $f(a) = f(b)$ and $a = b$ for all $a,b \in A$;
 \item If $a$ and $b$ are in $A$ and $f(a) = f(b)$, then $a = b$;
 \item If $a$ and $b$ are in $A$ and $a = b$, then $f(a) = f(b)$;
 \item If $a$ and $b$ are in $A$ and $f(a) \neq f(b)$, then $a \neq b$;
 \item If $a$ and $b$ are in $A$ and $a \neq b$, then $f(a) \neq f(b)$. 
 \end{enumerate}
 
\item Prove that the following functions are not injective. 

\begin{enumerate}
 \item $f \colon \mathbf{R}\to \mathbf{R}$, $f(x) = x^4 + x^2$;
 \item $f \colon \mathbf{R}\to \mathbf{R}$, $f(x) = x^3 + x^2$;
 \item $f \colon P(\mathbf{Z}) \to P(\mathbf{Z});\, f(S) = S \cap \{1,2\}$.

 \end{enumerate}

\item Prove that the following functions are injective.
  \begin{enumerate}
  \item $f\colon \mathbf{R}^2 \to \mathbf{R}^3;\, f(x,y) =  (x+y,x-y,x^2 + y^2)$.
  \item $f \colon \mathbf{R} \to \mathbf{R};\, f(x) = e^{x+1}$.
\item $f\colon \mathbf{R}\to \mathbf{R}, f(x) =
 \begin{cases}
 -x-1, & \text{if $x > 0$} \\
 x^2, & \text{if $x \leq 0$}.
 \end{cases}
 $
\end{enumerate}

\item Let $f\colon \{1,2,3,4,5\} \to \{1,2,3,4\}$ be a functon. Can $f$ be injective? Explain your answer.  

\item We say that a function $f\colon [a,b] \to \mathbf{R}$ is \textbf{decreasing} if for all $x_1,x_2 \in [a,b]$, if $x_1 < x_2$, then $f(x_1) > f(x_2)$.
 \begin{enumerate}
 \item Negate the definition of decreasing.
 \item Prove that a decreasing function is injective.
 \end{enumerate}
 
\item Let $A$ and $B$ be sets and let $X$ and $Y$ be subsets of $A$. Let $f \colon A \to B$ be an injective function. Prove that $f(X \cap Y) = f(X) \cap f(Y)$.

\item Let $A$ and $B$ be sets and let $W$ be a subset of $B$. Let $f \colon A \to B$ be a surjective function. Prove that $W \subseteq f(f^{-1}(W))$.

\end{enumerate}

%***************************************************************************************************************
%***************************************************************************************************************
\newpage
\section[10 (due \csname dateWeek10\endcsname): Composition of functions.]{Assignment 10}

% \warning 

\noindent\textbf{Topics}: Composition of functions.
\\

\noindent \textbf{Reading}: Section 5.1
\\


\noindent \textbf{Suggested problems (do not hand in; these are just for extra practice)}: \href{https://www.math.emory.edu/~dzb/teaching/250Fall2021/handouts/250-H12-compositions.pdf}{Handout 12}
\\

\noindent \textbf{Assignment 10, due Friday, \csname dateWeek10\endcsname, \HWdueTime via Gradescope}:
\\

\noindent For problems 1, 2, and 3, let $A, B$ and $C$ be sets and let $f \colon A \to B$ and $g \colon B \to C$ be functions. 
 \begin{enumerate}
 \item Prove or disprove: If $g\circ f$ is an injection, then $g$ is an injection.
 \item Prove or disprove: If $g\circ f$ is a surjection, then $f$ is a surjection.
 \item Prove or disprove: If $g\circ f$ is a surjection, then $g$ is a surjection.


\item Let $A$ and $B$ be sets and let $f \colon A \to B$ and $g \colon B \to A$ be functions. Prove that if $g \circ f$ and $f \circ g$ are bijective, then so are $f$ and $g$.

 \item Let $f \colon \R \to \R$ and $g \colon \R \to \R$ be functions. Suppose that $f$ and $g$ are both decreasing. Prove that $g \circ f$ is increasing.
 \end{enumerate}



%***************************************************************************************************************
%***************************************************************************************************************
\newpage
\section[ (November 17) Midterm]{Midterm (November 17)}

% \warning 

\noindent \textbf{Content}: The questions will all be either
\begin{enumerate}
\item homework problems,
\item suggested problems,
\item problems we worked in class, or
\item minor variations of one of these.
\end{enumerate}

Problems with very long proofs or that involved some unusual trick will not be on the exam.
\smallskip

You are allowed to use any previous problem from class or from the homework (e.g., ``$A \subseteq A \cup B$'') on the exam without reproving it, unless otherwise noted on the exam. (E.g., if I ask you to prove ``$A \subseteq A \cup B$'' on the exam, you will need to prove this using only the definition of subset and union, and I will remind you of this in the statement of the problem.)
\smallskip

A typical exam will have one or two questions from each week of the course. You can expect problems about following:
\begin{itemize}
\item Give definitions (e.g., subset, union, intersection, preimage, image)
\item Proofs about sets (including power sets)
\item Proofs about functions, images, preimages
\end{itemize}
There will be at least one definition, and around 4-5 problems involving proofs (possibly including ``prove or disprove'' problems).
\smallskip

% For sets: I will ask one problem, verbatim, from the homework or from class.
% \smallskip

For definitions, I want a definition, in prose (complete sentences), and I want ``just'' the definition, and not any additional facts about the definition. (E.g., if you give the definition of rational, do not include that a rational number can be written in reduced form; that is a fact about rational numbers not part of the definition of rational.)
\smallskip

There will be one problem verbatim from the second page of Handout 12 (on compositions).






 
%***************************************************************************************************************
%***************************************************************************************************************
\newpage
\section[11 (due \csname dateWeek11\endcsname): Inverse functions.]{Assignment 11}

% \warning 

\noindent \textbf{Thanksgiving break is Monday, November 20 through Friday, November 24};

\noindent There will no class those days.
\\

\noindent \noindent\textbf{Topics}: Inverse functions.
\\

\noindent \textbf{Reading}: Section 5.3
\\

\noindent \textbf{Suggested problems (do not hand in; these are just for extra practice)} \href{https://www.math.emory.edu/~dzb/teaching/250Fall2021/handouts/250-H13-inverses.pdf}{Handout 13}
\\

\noindent \textbf{Assignment 11, due Friday, \csname dateWeek11\endcsname, \HWdueTime via Gradescope}:
\begin{enumerate}
\item Define $f \colon \mathbf{R}-\{1\} \to \mathbf{R} - \{1\}$ by $f(x) = \frac{x+1}{x-1}$. Prove that $f$ is a bijection. Find a formula for the inverse $f^{-1}(x)$, and prove that it is correct.
\item Let $f \colon \mathbf{R} \to \mathbf{R}$ be the function $f(x) = x^3 + x$. Prove that $f$ is invertible without finding a formula for $f^{-1}$.
\item Let $A, B$ and $C$ be sets and let $f \colon A \to B$ and $g \colon B \to C$ be functions. Prove that if $f$ and $g$ are invertible, then so is $g \circ f$, and prove that $(g \circ f)^{-1} = f^{-1} \circ g^{-1}$.
\item Let $A$ and $B$ be sets and let $f \colon A \to B$ be a function. Suppose that $f$ has a \emph{left inverse} $g$; that is, suppose that there exists a function $g \colon B \to A$ such that $g \circ f = id_A$. Prove that $f$ is injective.
\item Let $f \colon A \to B$ be a bijection. Let $g \colon B \to A$ be the inverse of $f$. Prove that $g$ is also a bijection.
\item Let $A$ be a set and let $f \colon A \to A$ be a function. Prove that $f \circ f = id_A$ if and only if ``$f$ is invertible and $f = f^{-1}$''.

\item Let $f,g \colon A \to B$ be two functions, and let $h \colon B \to C$ be a function. 
  \begin{enumerate}
  \item Prove that if $h$ is injective and
    $h \circ f = h \circ g$ then $f = g$.
  \item Give an example where $h$ is not injective and     $h \circ f = h \circ g$ but $f \neq g$.
\end{enumerate}
  
\item Let $f \colon A \to B$ be a function, and let $g,h \colon B \to C$ be two functions. 
  \begin{enumerate}
  \item Prove that if $f$ is surjective and
    $g \circ f = h \circ f$ then $g = h$.
  \item Give an example where $f$ is not surjective and $g \circ f = h \circ f$ but $g \neq h$.
\end{enumerate}

\end{enumerate}

%***************************************************************************************************************
%***************************************************************************************************************
\newpage
\section[12 (due \csname dateWeek12\endcsname): Relations]{Assignment 12}

% \warning

\noindent\textbf{Topics}: Relations.
\\

\noindent \textbf{Reading}: Chapter 4.
\\

\noindent \textbf{Suggested problems (do not hand in; these are just for extra practice)}:
\href{https://www.math.emory.edu/~dzb/teaching/250Fall2021/handouts/250-H14-equivalence-relations.pdf}{Handout 14}
\\


\noindent \textbf{Assignment 12, due Friday, \csname dateWeek12\endcsname, via Gradescope}:
\begin{enumerate}
\item Define a relation $R$ on the set $\mathbb{R}$ as follows: we say that $x \sim y$ if $|x| = |y|$. (Recall that $|x|$ is the absolute value of $x$) Determine whether this relation is reflexive, symmetric, transitive (and in each case give a proof or disproof).

\item Let $A = \{1,2,3\}$ and define a relation on $A$ by $a \sim b$ if $a + b \neq 3$. Determine whether this relation is reflexive, symmetric, transitive (and in each case give a proof or disproof).

\item Define a relation on $\mathbf{Z}$ given by $a \sim b$ if $a-b$ is divisible by $3$.
 \begin{enumerate}
 \item Prove that this is an equivalence relation.
 \item What integers are in the equivalence class of 18? (No proof necessary.)
 \item What integers are in the equivalence class of 31? (No proof necessary.) 
 \item How many distinct equivalence classes are there? What are they? (No proof necessary.)
 \end{enumerate}
\item Define a relation on $\mathbf{Z}$ given by $a \sim b$ if $a^2-b^2$ is divisible by $4$.
 \begin{enumerate}
 \item Prove that this is an equivalence relation.
 \item How many distinct equivalence classes are there? What are they? (No proof necessary.)
 \end{enumerate} 
\item Let $A$ be a set, and let $P(A)$ be the power set of $A$. Assume that $A$ is not the empty set. Define a relation on $P(A)$ by $X \sim Y$ if $X \subseteq Y$. Is this relation reflexive, symmetric, and/or transitive? In each case give a proof, or disprove with a counterexample. (For a counterexample, give an example of $A$, $X$, and $Y$ that disproves the statement.)

\item Define a relation $R$ on the set $\mathbb{R}$ as follows: we say that $x \sim y$ if $x - y \in \mathbb{Q}$. Determine whether this relation is reflexive, symmetric, transitive (and in each case give a proof or disproof).
  
\item Define a relation $R$ on the set $\mathbb{Z}$ as follows: we say that $x \sim y$ if there exists a non-negative integer $n$ such that $x = 2^ny$. Determine whether this relation is reflexive, symmetric, transitive (and in each case give a proof or disproof).

\item 

  Define a relation $R$ on the set $\mathbb{Z}\times \left(\mathbb{Z} - \{0\}\right)$ as follows: we say that $(a,b) \sim (c,d)$ if $ad = bc$. 
 \begin{enumerate}
 \item Prove that this is an equivalence relation.
 \item What pairs $(c,d)$ are in the equivalence class of $(1,1)$? (No proof necessary.)
 \item What pairs $(c,d)$ are in the equivalence class of $(1,2)$? (No proof necessary.)   
 \item How many distinct equivalence classes are there? What are they? (No proof necessary.)
 \end{enumerate} 
  
\end{enumerate}


% \noindent \textbf{Please note the unusual due date of the homework assignment}.



%***************************************************************************************************************
%***************************************************************************************************************
\newpage
\section[13 (Not to be handed in): Binary Operations and Bijections/Countability]{Assignment 13}

% \warning

\noindent\textbf{Topics}: Binary Operations and Bijections/Countability
\\

\noindent \textbf{Reading}: Chapter 4; Chapter 8
% Section 4.2, p.~139-144 (stop at the proof of Theorem 4.2.6)
\\



% \noindent \textbf{Suggested problems (do not hand in; these are just for extra practice)}:
% TODO  
% \\


% \noindent \textbf{Assignment 13, due Friday, \csname dateWeek13\endcsname, via Gradescope}:
\noindent \textbf{\sout{Assignment} Topic 13, not to be handed in}: Consider the final two lectures ``bonus content'': there are a few topics that are very helpful to know in upper level courses, but which we were only able to briefly cover in class. I will not assign any graded work on these topics, and will not test you on these topics on the exam. If I were to assign some problems, I would assign the ones below (and some additional problems). 

% one or two problems from this assignment might appear verbatim on the final. There would not be time for the graders to grade this assignment before the final; so instead, I am happy to explain in full detail how to do any problem from this assignment.




\begin{enumerate}
\item Let $S$ be one of the following sets. Give an example of a bijection from $S$ to $\mathbb{Z}$. (No proof is necessary.)
  \begin{enumerate}
  \item $S = n\mathbb{Z}$ (for some fixed positive integer $n$).
  \item $S = \mathbb{Z} \times \{0\}$.    
  \item $S = \mathbb{Z} \times \{-1,1\}$.
  \item $S = \sin^{-1}(\{0\})$.
  \end{enumerate}

% \item Give an example of a bijection from $\mathbb{N}$ to ..
  
\item  Let $g\colon \text{Fun}(S, \{0,1\})  \to P(S)$ be the function $f
\mapsto f^{-1}(\{0\})$. Prove that $f$ is a bijection.\\

\item  As in class, let $P_{\text{BD}}(S)$ be the set of all finite
subsets of $S$, i.e., 
\[
P_{\text{BD}}(S) = \{ A \subset S \,|\, |A| < \infty \}
\]
and for $i \in \Z_{ >  0}$ let 
\[
P_{i}(S) = \{ A \subset S \,|\, |A| \leq i \}.
\]
\begin{itemize}
\item [(a)] Prove that
\[
P_{\text{BD}}(S) = \bigcup_{i \in \Z_{> 0}} P_i(S).
\]
\item [(b)] Prove that the map
\[
 S^i \to P_i(S) 
\]
given by 
\[
(a_1, \ldots, a_i) \mapsto \{a_1,\ldots,a_i\}
\]
is a surjection.
\item [(c)] Prove that $P_{\text{BD}}(\Z)$ is countable.\\
\end{itemize}

\item  Prove that $|S| \neq |\text{Fun}(S, \{0,1\})|$ by the same
technique as the proof from class that $|S| \neq |P(S)|$. (In other
words, proceed by contradiction, assuming that there is a bijection
$f\colon S \to \text{Fun}(S, \{0,1\})$ and then construct some $g \in
\text{Fun}(S, \{0,1\})$ which is not in the image of $f$.)
\end{enumerate}


 % \noindent \textbf{Please note the unusual nature of the homework assignment}.



%***************************************************************************************************************
%***************************************************************************************************************
\newpage
\section[Final Exam]{Final Exam}

% \warning

\noindent The \textbf{Final exam} is \textbf{December 18}, 2-5pm, in SMUD 204.
\\

\noindent The \textbf{last day of class} is Wednesday, December 13. 
\\

\noindent There will be \textbf{office hours} before the exam. I will send out a survey to find a time that works for everyone who is planning to attend.% 10:30-11:30 and 6-7, via Zoom. 
\\

The final exam will be comprehensive.
\\
% and will only cover content introduced after the midterm. Still, while I won't give you problems that are ``just'' about induction, contradiction, negations, etc.~(so for example, I will not ask any irrationality questions) you will still need to use those techniques in some of your proofs.
% \\

% The final exam will not be comprehensive, and will only cover content introduced after the midterm. Still, while I won't give you problems that are ``just'' about induction, contradiction, negations, etc.~(so for example, I will not ask any irrationality questions) you will still need to use those techniques in some of your proofs.
% \\


The exam will be, roughly 8-10 questions, with multiple parts. Some questions will be ``prove or disprove''. For disproofs, please write out a counterexample as your disproof.
\\

A typical exam will have one or two questions from each week of the course. You can expect a subset of the following:
\begin{itemize}
\item Negations
\item Give definitions (e.g., divides, rational, subset)
\item Direct proofs
\item Proof by contrapositive
\item Divisibility problems
\item Contradiction
\item Induction
\item Proofs with sets. 
\item Images
\item preimages
\item Injectivity 
\item Surjectivity
\item Compositions
\item Invertibility
\item Relations
\item \sout{Countability}
\item Problems from handouts 9-14
\end{itemize}


% Typical exams:
%\ques{30} No definitions.
% Negate
% Give definitions
% Proof by Induction
% Contradiction
% Contrapositive
% Set Identities
% Abstract functions
% Injectivity 
% Surjectivity
% Invertibility
% Relations
% Binary Operations
% Variants of problems from HO 12-14
% Problems from HO 15-16


 \end{document}

Other potential problems
 \begin{enumerate}
 \item Let $A$ and $B$ be sets. We define the \textbf{symmetric difference of} $A$ and $B$ to be
\[
A \Delta B := \{x | x \in A \cup B \text{ and } x \not \in A \cap B\}.
\]

\begin{enumerate}
\item Prove or disprove: $f(A \Delta B) \subset f(A) \Delta f(B)$.
\item Prove or disprove: $f(A) \Delta f(B) \subset f(A \Delta B)$.
\end{enumerate}
\end{enumerate}