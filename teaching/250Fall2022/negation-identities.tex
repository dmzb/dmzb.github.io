\documentclass[12pt, reqno]{amsart}

\textwidth=7in
\textheight=9.5in
\topmargin=0in
\headheight=0in
\headsep=.5in
\hoffset=-.85in

\begin{document}

\title[Math 250: useful  identities]{Math 250: useful identities}\maketitle

Here are some basic identities.
\begin{enumerate}
\item $P \wedge Q = Q \wedge P$
\item $P \vee Q = Q \vee P$
\item $(P \wedge Q) \wedge R = P \wedge (Q \wedge R) = P \wedge Q \wedge R$
\item $(P \vee Q) \vee R = P \vee (Q \vee R) = P \vee Q \vee R$  
\end{enumerate}


Here are some useful identities.

\begin{enumerate}
\item $\neg(P \wedge Q) = \neg P \vee \neg Q$
\item $\neg(P \vee Q) = \neg P \wedge \neg Q$
\item $\neg(\neg P) = P$
\item $P \vee (Q \wedge R) = (P \vee Q) \wedge (P \vee R)$
\item $P \wedge (Q \vee R) = (P \wedge Q) \vee (P \wedge R)$
\item $\neg (P \Rightarrow Q) = P \wedge \neg Q$
\item $\neg (\forall x, P(x)) = \exists x$  such that $\neg P(x)$
\item $\neg (\exists x$ such that $P(x)) = \forall x, \neg P(x)$

\end{enumerate}
\vspace{10pt}

We can combine these to negate more complicated statements

\begin{enumerate}
\item $\neg(P \Rightarrow (Q \vee R)) = $ 
\item [] $P \wedge \neg (Q \vee R)) = $ 
\item [] $P \wedge \neg Q \wedge \neg R$ 
\item []
\item If $1 = 0$ and $2 + 2 = 5$, then the sky is blue and kittens are cute
\item [] If (P and Q) then (R and T)
\item [] 
\item [] Its negation:
\item [] (P and Q) and not (R and T)
\item [] ($1 = 0$ and $2 + 2  = 5$) and (the sky is not blue or kittens are not cute)
\item []
\item $\neg Q \Rightarrow \neg P$
\item [] $\neg (\neg Q \Rightarrow \neg P)$
\item [] $\neg Q \wedge \neg(\neg P)$
\item [] $\neg Q \wedge P$
\end{enumerate}

\vspace{10pt}

This last example is called the contrapositive, and is a useful proof technique! (Try it on your homework.)

\begin{enumerate}
\item $(P \Rightarrow Q) = (\neg Q \Rightarrow \neg P)$ because they have the same negation.
\end{enumerate}



\end{document}

